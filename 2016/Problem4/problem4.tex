\documentclass{article}

\usepackage{amsmath}
\usepackage{physics}

\usepackage[utf8]{inputenc}

\title{Problem Sheet 4 \\
\large{FYS3110}}

\author{Sebastian G. Winther-Larsen}

\date{\today}

\begin{document}

\maketitle

The hamiltonian of particle with mass $m$ in a one-dimensional oscillator potetial having a characteristic frequency $\omega$ is
\begin{equation}
\hat{H}=\frac{1}{2m}\hat{p}^2+\frac{1}{2}m\omega^2\hat{X}^2
\end{equation}
The ladder operators for the harmonic oscillator potential are
\begin{align}
\hat{a}&=\frac{1}{\sqrt{2\hbar m\omega}}(m\omega\hat{X}+i\hat{P})
&\text{(lowering operator)} \label{eq:ladderdown} \\
\hat{a}^{\dagger}&=\frac{1}{\sqrt{2\hbar m\omega}}(m\omega\hat{X}-i\hat{P})
&\text{(raising operator)} \label{eq:ladderup}
\end{align}

\section*{Problem 4.1}

\subsection*{a)}

I want to find an expression for $\hat{X}$ in terms of $\hat{a}_{nm}$ and $\hat{a}^{\dagger}_{nm}$. This can be done by first rewriting equation \ref{eq:ladderdown}
\begin{align*}
\hat{a}=\frac{1}{\sqrt{2\hbar m\omega}}(m\omega\hat{X}+i\hat{P})
= \sqrt{\frac{m\omega}{2\hbar}}\hat{X}+\frac{i}{\sqrt{2\hbar m \omega}}\hat{P} \\
\rightarrow \hat{X} = \sqrt{\frac{2\hbar}{m\omega}}\hat{a}-\sqrt{\frac{2\hbar}{m\omega}}\frac{i}{\sqrt{2\hbar m\omega}}\hat{P}, 
\end{align*}
and then equation \ref{eq:ladderup}
\begin{align*}
\hat{a}^{\dagger} = \frac{1}{\sqrt{2\hbar m\omega}}(m\omega\hat{X}-i\hat{P}) = \sqrt{\frac{2\hbar}{m\omega}}\hat{X}+\frac{i}{2\hbar m\omega}\hat{P} \\
\rightarrow \hat{P}= \frac{2\hbar m \omega}{i}\sqrt{\frac{m\omega}{2\hbar}}\hat{X}-\frac{2\hbar m\omega}{i}\hat{a}^{\dagger}.
\end{align*}
Now putting the latter equation into the former yields
\begin{align*}
\hat{X}&=\sqrt{\frac{2\hbar}{m\omega}}\hat{a}-\sqrt{\frac{2\hbar}{m\omega}}\frac{i}{2\hbar m\omega}\left(\frac{2\hbar m\omega}{i}\sqrt{\frac{m\omega}{2\hbar}}\hat{X}-\frac{2\hbar m\omega}{i}\hat{a} \right) \\
&=\sqrt{\frac{2\hbar}{m\omega}}\hat{a}+\sqrt{\frac{2\hbar}{m\omega}}\hat{a}^{\dagger}-\sqrt{\frac{2\hbar}{m\omega}}\sqrt{\frac{m\omega}{2\hbar}}\hat{x},
\end{align*}
which simplifies to
\begin{equation}
\label{eq:xoperator}
\hat{X}=\sqrt{\frac{\hbar}{2m\omega}}(\hat{a}+\hat{a}^{\dagger})
\end{equation}

It will be necessary to know what $[\hat{a},\hat{H}]$ is. This is easiest to compute if one knows how $\hat{a}$ and $\hat{a}^{\dagger}$ is related to $\hat{H}$. By looking at the expressions for $\hat{a}$ and $\hat{a}^{\dagger}$ one is tempted to compute the following
\begin{equation*}
\hat{a}\hat{a}^{\dagger}=\frac{m\omega}{2\hbar}\hat{X}^2+\frac{1}{2m\omega\hbar}\hat{P}^2+\frac{i}{2\hbar}[\hat{X},\hat{P}],
\end{equation*}
where $[X,P]=i\hbar$, which follows from $\hat{X}\rightarrow x$ and $\hat{P}\rightarrow i\hbar (d/dx)$, but is independent of basis. So we see that
\begin{equation}
\label{eq:hamilton}
\hat{H} = \hbar\omega(\hat{a}\hat{a}^{\dagger}+\frac{1}{2}),
\end{equation}
or alternatively that $\hat{a}\hat{a}^{\dagger}=\frac{\hat{H}}{\hbar \omega}-\frac{1}{2}$.
This combined ladder operator can be referred to by a new name $\hat{a}\hat{a}^{\dagger}=\hat{N}$, so that equation \ref{eq:hamilton} becomes
\begin{equation}
\hat{H} = \hbar\omega(\hat{N}+\frac{1}{2}),
\end{equation}
Thus, we have the energy eigenbasis which satisfy
\begin{equation}
\hat{H}\ket{n}=\hbar\omega(n-\frac{1}{2})\ket{n}, \text{ for } i \in 0, 1, 2 \dots
\end{equation}

We now get
\begin{equation}
\hat{a}\ket{n} = C_n\ket{n},
\end{equation}
where $C_n$ is a constant which can be found the following way
\begin{align*}
\bra{n}\hat{a}^{\dagger}\hat{a}\ket{n} = & \abs{C_n}^2\braket{n-1}\\
\rightarrow & \abs{C_n}=\sqrt{n}=C_n,
\end{align*}
by choosing the phase to be zero\footnote{Actually, $C_n=\sqrt{n}e^{i\phi}$ where $\phi$ is arbitrary, but is is conventional to set $\phi=0$.}. We land at
\begin{equation}
\hat{a}\ket{n}=\sqrt{n}\ket{n-1}
\end{equation}
Similarly for $\hat{a}^{\dagger}$
\begin{equation}
\hat{a}^{\dagger}\ket{n} = D_n\ket{n+1}
\end{equation}
\begin{align*}
\bra{n}\hat{a}\hat{a}^{\dagger}\ket{n}=&\abs{D_n}^2\braket{n+1} \\
\rightarrow & \abs{D_n}=\sqrt{n+1}=D_n
\end{align*}
\begin{equation}
\hat{a}^{\dagger}\ket{n}=\sqrt{n+1}\ket{n+1}
\end{equation}

Now to compute the matrix elements for $\hat{a}$ and $\hat{a}^{\dagger}$
\begin{align}
\bra{m}\hat{a}\ket{n}=\sqrt{n}\braket{m}{n-1}=&\sqrt{n}\delta_{m,n-1} \\
\bra{m}\hat{a}^{\dagger}\ket{n}=\sqrt{n-1}\braket{m}{n+1}=&\sqrt{n-1}\delta_{m,n+1}.
\end{align}
The actual matrices for these particular operators will look something like this\footnote{I am perfectly aware that $\sqrt{1}=1$ but I will keep the root symbol here in order to underline the symmetry.}
\begin{align}
a &\leftrightarrow \begin{bmatrix}
0 & \sqrt{1} & 0 & 0 & \dots \\
0 & 0 & \sqrt{2} & 0 &  \\
0 & 0 & 0 & \sqrt{3} & \\
\vdots & & & & \ddots 
\end{bmatrix} \\
a^{\dagger} &\leftrightarrow \begin{bmatrix}
0 & 0 & 0 & \dots \\
\sqrt{1} & 0 & 0 & \\
0 & \sqrt{2} & 0 & \\
0 & 0 & \sqrt{3} & \\
\vdots & & & \ddots 
\end{bmatrix}.
\end{align}

Finding the matrix representation of $\hat{X}$ is now an easy matter of employing equation \ref{eq:xoperator}
\begin{equation}
\label{eq:xmatrixelements}
X \leftrightarrow  \sqrt{\frac{\hbar}{2m\omega}}\begin{bmatrix}
0 & \sqrt{1} & 0 & 0 & \dots \\
\sqrt{1} & 0 & \sqrt{2} & 0 &  \\
0 & \sqrt{2} & 0 & \sqrt{3} & \\
0 & 0 & \sqrt{3} & 0 \\
\vdots & & & & \ddots 
\end{bmatrix},
\end{equation}
while an algebraic expression will  be (also employing equation \ref{eq:xoperator})
\begin{align}
\bra{m}X\ket{n} &= \sqrt{\frac{\hbar}{2m\omega}}(\sqrt{n+1}\braket{m}{n+1}+\sqrt{n}\braket{m}{n-1}) \nonumber \\
&= \sqrt{\frac{\hbar}{2m\omega}}(\sqrt{n+1}\delta_{m,n+1}-\sqrt{n}\delta_{m,n-1}). \label{eq:xmatrixelements2}
\end{align}

\subsection*{b)}
Let $\ket{\psi(0)}=\sum_{x=0}^n c_n\ket{n}$. It follows that
\begin{align*}
\braket{\psi(0)} &= \sum_n \abs{c_n}^2 \braket{n} + 2\sum_{n \neq n'} c_nc_{n'} \braket{n}{n'} \\
 &= \sum_n \abs{c_n}^2 \delta_{n,n} + 2\sum_{n \neq n'} c_nc_{n'} \delta_{n,n'}, 
\end{align*}
where the first Kronecker delta will always be $1$ and the second Kronecker delta will always be $0$. The condition on the $c_n$'s for $\ket{\psi(0)}$ to have unit norm is therefore
\begin{equation}
\sum_n \abs{c_n}^2 = 1
\end{equation}

Letting the lowering operator work on the ground state outputs the ground state energy, $\hat{a}\ket{0}=0$, here arbitrarily set to zero. Letting, $\ket{0} \mapsto \psi_0(x)$ and $\hat{a} \mapsto \frac{1}{\sqrt{2\hbar m \omega}}(m\omega \hat{X} + i\hat{P})$, one can expand upon this idea to find an expression for the ground state
\begin{align*}
\hat{a}\ket{0}&=\frac{1}{\sqrt{2\hbar m\omega}}(m\omega \hat{X} + i\hat{P})\psi_0(x) = \frac{1}{\sqrt{2\hbar m\omega}}(m\omega x + i\frac{\hbar}{i}\frac{d}{d x})\psi_0(x) = 0\\
&\rightarrow \left(m\omega x + \hbar\frac{d}{d x}\right)\psi_0(x)=0 \rightarrow \frac{d}{d x} = -\frac{m\omega x}{\hbar} \psi_0(x) \\
&\rightarrow \frac{d \psi_0(x)}{\psi_0}=-\frac{m\omega x}{\hbar} d x \rightarrow \psi_0(x) = C_0e^{-m\omega x^2/2\hbar}
\end{align*}
Normalizing, denoting $\alpha = \frac{m\omega}{\hbar}$
\begin{align*}
\braket{\psi_0(x)} = \abs{C_0}^2\int_{-\infty}^{\infty}e^{-\alpha x^2}dx = \abs{C_0}^2\sqrt{\frac{\pi}{\alpha}}=1,
\end{align*}
which gives
\begin{equation}
C_0=\left(\frac{\pi\hbar}{m\omega}\right)^{\frac{1}{4}}
\end{equation}

The full wave equation, $\Psi(x,y)$, is a product of two parts, the time-independent equation, $\psi(x)$, and the time-dependent equation $\phi(t)=ce^{-i\ E_n t/\hbar}$
\begin{equation}
\Psi(x,t)=\sum_{n=1}^{\infty}c_n\psi_n(x)e^{i\ E_n t/\hbar}
\end{equation}

Alternatively, one can express the wave function with Dirac notation using $\psi_n=\braket{x}{n}$
\begin{equation}
\psi_1(x) = \bra{x}\hat{a}^{\dagger}\ket{0}
\end{equation}
For a general expression one needs some general function for the eigenstates $\ket{n}$. These can be expressed as in terms of the ground state $\ket{0}$
\begin{equation*}
\hat{a}^{\dagger}\ket{n} = \sqrt{n+1}\ket{n+1} \rightarrow \ket{n}=\frac{\hat{a}^{\dagger}}{\sqrt{n}}\ket{n-1} = \frac{(\hat{a}^{\dagger})^2}{\sqrt{n(n-1)}}\ket{n-1} \dots
\end{equation*}
thus giving the expression
\begin{equation}
\ket{n} = \frac{(\hat{a}^{\dagger})^n}{\sqrt{n!}}\ket{0}
\end{equation}
All the ingredients are there to write down the complete time-dependent wave function
\begin{equation}
\Psi_n(x,t)=\bra{x}\phi_n(t)\ket{n}=\bra{x}e^{-\frac{i E_n }{\hbar}t}\frac{(\hat{a}^{\dagger})^n}{\sqrt{n!}}\ket{0}
\end{equation}

\subsection*{c)}

When computing the expected values of a harmonic oscillator Schrödinger wave function the time-dependency parts will equate to one, given normality.
\begin{equation*}
\psi*(t)\psi(t)=e^{\frac{iE_nt}{\hbar}}e^{-\frac{iE_nt}{\hbar}}=e^0=1
\end{equation*}

Now the computation of the expected values are fairly straight-forward
\begin{align*}
\ev{\hat{X}}_{nm}&=\bra{\psi_n}\hat{X}\ket{\psi_m}=\sqrt{\frac{\hbar}{2m\omega}}\bra{\psi_n}(\hat{a}+\hat{a}^{\dagger})\ket{\psi_m} \\
&=\sqrt{\frac{\hbar}{2m\omega}}(\bra{\psi_n}\sqrt{m}\ket{\psi_{m-1}}+\bra{\psi_n}\sqrt{m+1}\ket{\psi_{m+1}}) \\
&= \sqrt{\frac{\hbar}{2m\omega}}(\sqrt{m}\bra{\psi_n}\ket{\psi_{m-1}}+\sqrt{m+1}\bra{\psi_n}\ket{\psi_{m+1}}) \\
&= \sqrt{\frac{\hbar}{2m\omega}}(\sqrt{m}\delta_{n,m-1}+\sqrt{m+1}\delta_{n,m+1}) 
= \begin{cases}
	\sqrt{\frac{\hbar (m+1)}{2m\omega}}, & n=m+1 \\
	\sqrt{\frac{\hbar m}{2m\omega}}, & n=m-1\\
	0, & \text{otherwise}
\end{cases}
\end{align*}
which has a similar symmetry to the matrix elements in equations \ref{eq:xmatrixelements} and \ref{eq:xmatrixelements2}. In fact, it is the exact same thing. 
\begin{align*}
\ev{\hat{H}}_{nm}&=\bra{\psi_n}\hat{H}\ket{\psi_m}=\bra{\psi_n}\hbar\omega\left(\hat{a}\hat{a}^{\dagger}+\frac{1}{2}\right)\ket{\psi_m} \\
&= \hbar\omega \bra{\psi_n}\hat{a}\hat{a}^{\dagger}\ket{\psi_m}+\frac{\hbar\omega}{2}\braket{\psi_n}{\psi_m}\\
&=\hbar\omega m\braket{\psi_n}{\psi_m}+\frac{\hbar\omega}{2}\braket{\psi_n}{\psi_m} = \hbar\omega\left(m+\frac{1}{2}\right) \delta_{n,m}
\end{align*}

\section*{4.2}

In three dimensions the Schrödinger equation for the harmonic oscillator using cartesian coordinates is
\begin{equation}
\label{eq:3dschr}
-\frac{\hbar}{2m}\laplacian \psi + \frac{1}{2}(x^2+y^2+z^2)\psi = E\psi
\end{equation}
This equation can be solved by employing esperation of variables, such that
\begin{equation}
\label{eq:3dsep}
\psi(x,y,z)=\xi(x)\upsilon(y)\zeta(z)
\end{equation}
Dividing equation \ref{eq:3dschr} by equation \ref{eq:3dsep} yields
\begin{equation}
-\frac{\hbar^2}{2m}\frac{1}{\xi}\frac{d^2\xi}{dx} + \frac{1}{2}m\omega^2x^2
-\frac{\hbar^2}{2m}\frac{1}{\upsilon}\frac{d^2\upsilon}{dy} + \frac{1}{2}m\omega^2x^2
-\frac{\hbar^2}{2m}\frac{1}{\zeta}\frac{d^2\zeta}{dx} + \frac{1}{2}m\omega^2x^2
=E
\end{equation}
The reasoning behind such an operation, which leaves a seemingly untidy expression, is that we now have three seperate groups, each of which depends only on one of the coordinate variables. The sum of all these groups equals the constant $E$, and one can argue that each of the groups on its own must therefore be equalt to a constant. Some may find this assumption a bit farfetched, but it makes the problem vastly more simple, by reducing the equation dependent on three variables to three equations each dependent on one variable. From the one-dimensional case we can furhtermore assume that each of the dimensions will contribute $\hbar\omega/(n+\frac{1}{2})$ towards the totat energy $E$. The energy will still have the same discretization, but wil have a ground sate three times as ``energetic'' as in the one-dimensional case, $E_0=\frac{3}{2}\hbar\omega$. Energy level $n$ will have an energy of
\begin{equation}
E_n=\left(n+\frac{3}{2}\right)\hbar\omega.
\end{equation}

When going from the ground state, where $E_0=\frac{3}{2}\hbar\omega$, to the first state, where $E_1=\frac{5}{2}\hbar\omega$, the increase in energy may come from either of the three dimensions as $n=n_x+n_y+n_z$. This implies that all states, except the ground state, must be degenerate. One can alternatively write the energy of state $n$ as
\begin{equation}
E(n_x+n_y+n_z)=(n_x+n_y+n_z+\frac{3}{2})\hbar\omega.
\end{equation}

The number of degenerative states is equal to the number of ways one can write an integer $n$ as a sum of three non-negative integers. For a given $n$ choose $n_x=0,1,2,3,...,n$, then choose $n_y=0,1,2,3,...,n-n_x$. Now $n_z$ is determined by the two previous choices $n-n_x-n_y$. Ergo, for each $n_x$ there are $n-_nx+1$ possible choices for $n_y$. This gives
\begin{align*}
d(n)&=\sum_{n_x=0}^n (n+1-n_x) = \sum_{n_x=0}^n n  +\sum_{n_x=0}^n 1 - \sum_{n_x=0}^n n_x \\
&= (n+1)n-\frac{1}{2}n(n+1)+n+1 = (n+1)\left(n-\frac{n}{2}+1 \right) \\
&= (n+1)\left(\frac{n}{2} +1\right) = \frac{1}{2}(n+1)(n+1)
\end{align*}
This result amounts to the dimensionality of a symmetric representation of the special unitary group of degree 3\footnote{Fradkin, D. M. (1965). Three-dimensional isotropic harmonic oscillator and SU3. American Journal of Physics, 33(3), 207-211.}

\subsection*{4.3}
Changing from position space to momentum space is done by way of Fourier transform
\begin{equation}
\phi(p)=\frac{1}{\sqrt{2\pi\hbar}}\int_{-\infty}^{\infty}\psi(x)e^{-ipx/\hbar}dx
\end{equation}

In momentum space $\hat{P}\mapsto p$ and $\hat{X}\mapsto i\hbar\frac{d}{dp}$. The lowering operator becomes
\begin{equation}
\hat{a} \mapsto \frac{1}{\sqrt{2\hbar m\omega}}\left(m\omega \hat{X}+\hat{P} \right)=\frac{1}{\sqrt{2\hbar m\omega}}\left(m\omega i\hbar\frac{d}{dp}+p \right)
\end{equation}
The ground state in momentum space can be found by looking at $\hat{a}\ket{0}=0$, where $\ket{0} \mapsto \phi_0(p)$.
\begin{align*}
\hat{a}\ket{0} &= \frac{1}{\sqrt{2\hbar m\omega}}\left(m\omega i\hbar\frac{d}{dp}+p \right)\phi_0(p)=0 \\
&\rightarrow i \sqrt{\frac{m\omega\hbar}{2}}\frac{d\phi_0}{dp}=-\frac{ip\phi_0}{\sqrt{2\hbar m \omega}} \\
&\rightarrow \frac{1}{\phi_0}d\phi_0=-\frac{1}{m\hbar\omega}pdp
\end{align*}
Taking the integral on both sides yields
\begin{equation}
\phi_0(p)=c_0e^{-p^2/2m\hbar\omega}
\end{equation}

The first excited state can be found by applying $\hat{a}^{\dagger}$
\begin{align*}
\hat{a}^{\dagger}\phi_0 &=\frac{c_0}{\sqrt{2\hbar m\omega}}\left(m\omega i\hbar\frac{d}{dp} -ip\right)e^{-p^2/2m\hbar\omega} \\
&= -\frac{c_0}{\sqrt{2\hbar m\omega}}\left(\frac{-im\hbar\omega}{2m\hbar\omega}2pe^{-p^2/2m\hbar\omega} -ipe^{-p^2/2m\hbar\omega}\right) \\
&= -i\frac{c_0}{2\hbar m\omega}pe^{-p^2/2m\hbar\omega},
\end{align*}
which, interestingly enough, is an imaginary number..

\end{document}