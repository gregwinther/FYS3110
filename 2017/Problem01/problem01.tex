\documentclass[11pt]{article}

\usepackage{amsmath, amssymb}
\usepackage{physics}

\usepackage[utf8]{inputenc}

\title{Problem set 1 \\ \large{FYS3110}}

\author{Sebastian G. Winther-Larsen}

\date{\today}

\begin{document}

\maketitle

\section*{Problem 1.1(L)}

The complex inner product $\braket{u}{v}$ is linear in its \emph{second} factor, which means: Given $\ket{v} = \alpha\ket{v_1} + \beta\ket{v_2}$, where $\alpha, \beta \in \mathbb{C}$, then $\braket{u}{v} = \alpha\braket{u}{v_1} + \beta\braket{u}{v_2}$. However, the complex inner product is \emph{not} linear in its \emph{first} factor.

If $\ket{u} = \alpha\ket{u_1} + \beta\ket{u_2}$ then $\bra{u} = \alpha^*\bra{u_1} + \beta^*\bra{u_2}$, now
\begin{equation}
\braket{u}{w} = \alpha^*\braket{u_1}{w} +  \beta^*\braket{u_2}{w},
\end{equation}
for an arbitrary $\ket{w}$. This leads one to conclude that the complex inner product is \emph{antilinear} in the first factor.

\section*{Problem 1.2(L)}

The following property holds for the inner product of any two vectors $\ket{\alpha}$ and $\ket{\beta}$
\begin{equation}
\braket{\beta}{\alpha} = \braket{\alpha}{\beta}^*.
\end{equation}


\end{document}	