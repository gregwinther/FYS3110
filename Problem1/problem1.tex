\documentclass{article}
\usepackage{amsmath}
\usepackage{physics}

\title{Problem Sheet 1 \\
\large{FYS3110}}
\author{Sebastian G. Winther-Larsen (sebastwi)}


\begin{document}

\maketitle

\subsection*{Problem 1.1}

\hspace{1cm}

a)

\begin{align*}
\ket{\Psi} 		&= c(\sqrt{5}\ket{0}+i\ket{1})\\
\braket{\Psi} 	&= c^*(\sqrt{5}\bra{0}-i\bra{1}) c(\sqrt{5}\ket{0}+i\ket{1})\\
				&= c^*c(5\braket{0}+\sqrt{5}i\braket{0}{1}-\sqrt{5}\braket{1}{0}+\braket{1})\\
				&= \abs{c}(5+1) \rightarrow \abs{c}=\frac{1}{6}
\end{align*}

\hspace{1cm}

b)

\hspace{1cm}

The following representation is used for the basis vectors
\begin{align*}
\ket{0} &= \begin{pmatrix} 1 \\ 0 \end{pmatrix}\\
\ket{1} &= \begin{pmatrix} 0 \\ 1 \end{pmatrix}
\end{align*}

it follows that

\begin{align*}
\ket{\Psi} 	&= c(\sqrt{5}\ket{0}+i\ket{1}) \\
			&= c\left(
				\sqrt{5} \begin{pmatrix} 1 \\ 0 \end{pmatrix}
				i \begin{pmatrix} 0 \\ 1 \end{pmatrix}
				\right) \\
			&= c \begin{pmatrix} \sqrt{5} \\ i \end{pmatrix} \\[10pt]
\end{align*}

\begin{align*}
\bra{\Psi} 	&= c^*(\sqrt{5}\bra{0}-i\bra{1}) \\
			&= c^*( \sqrt{5} ( 1 \quad 0 ) -i ( 0 \quad 1 )) \\
			&= c^* ( \sqrt{5} \quad  -i ) \\[10pt]
\braket{\Psi} &= \abs{c}6 \rightarrow \abs{c}=\frac{1}{6}
\end{align*}

Furthermore, $\hat{A}$ is an operator defined thusly
\begin{equation*}
\hat{A}\ket{0} = -i\ket{1}, \quad \hat{A}\ket{1} = i\ket{0}
\end{equation*}


It is quite easy to see that 
\begin{equation*}
\hat{A} = \begin{pmatrix} 0 & i \\ -i & 0 \end{pmatrix}
\end{equation*}

$\hat{A}$ applied to $\ket{0}$:
\begin{equation*}
\hat{A}\ket{0} =
\begin{pmatrix} 0 & i \\ -i & 0 \end{pmatrix} \begin{pmatrix} 1 \\ 0 \end{pmatrix} = \begin{pmatrix} 0 \\ -i \end{pmatrix} = -i \begin{pmatrix} 0 \\ 1 \end{pmatrix} = -i\ket{1}
\end{equation*}


and $\hat{A}$ applied to $\ket{1}$:
\begin{equation*}
\hat{A}\ket{1} =
\begin{pmatrix} 0 & i \\ -i & 0 \end{pmatrix} \begin{pmatrix} 0 \\ 1 \end{pmatrix} = \begin{pmatrix} i \\ 0 \end{pmatrix} = i \begin{pmatrix} 1 \\ 0 \end{pmatrix} = i\ket{0}
\end{equation*}

\hspace{1cm}

c)

\hspace{1cm}

There are now two ways to calculate $\bra{\Psi}\hat{A}\ket{\Psi}$

\begin{align*}
\bra{\Psi}\hat{A}\ket{\Psi} &=
c^*\begin{pmatrix}\sqrt{5} & -i \end{pmatrix}
\begin{pmatrix} 0 & i \\ -i & 0 \end{pmatrix}
c\begin{pmatrix} \sqrt{5} \\ i \end{pmatrix} \\ &=
\abs{c} \begin{pmatrix} \sqrt{5} & -i \end{pmatrix}
\begin{pmatrix} -1 \\ -\sqrt{5}i \end{pmatrix} \\ &=
\frac{1}{6}(-\sqrt{5} - \sqrt{5}) = -\frac{\sqrt{5}}{3}
\end{align*}

\begin{align*}
\bra{\Psi}\hat{A}\ket{\Psi} &=
c^*(\sqrt{5}\bra{0}í\bra{1})\hat{A}c(\sqrt{5}\ket{0}+i\ket{1}) \\ &=
\abs{c}(\sqrt{5}\bra{0}-i\bra{1})(-\sqrt{5}i\ket{1}-\ket{0}) \\ &=
\frac{1}{6}(-5i\braket{0}{1}-\sqrt{5}\braket{0}{0}-\sqrt{5}\braket{1}{1}+i\braket{1}{0}) \\ &=
\frac{1}{6}(-2\sqrt{5}) = -\frac{\sqrt{5}}{3}
\end{align*}

\hspace{1cm}

\subsection*{Problem 1.2}

\hspace{1cm}

A complex matrix is given as

\begin{equation*}
U = \begin{pmatrix} a & b \\ c & d \end{pmatrix}
\end{equation*}

where $a, b, c, d$ are complex numbers.

\hspace{1cm}

a)

\hspace{0.5cm}

\begin{align*}
U^T &= \begin{pmatrix} a & c \\ b & d \end{pmatrix} \\
U^{\dagger} &= \begin{pmatrix} a^* & b^* \\ c^* & d^* \end{pmatrix}^T
=\begin{pmatrix} a^* & c^* \\ b^* & d^* \end{pmatrix}
\end{align*}

\vspace{1cm}

b)

\vspace{0.5cm}

$U$ is \emph{Hermittian} if $U=U^{\dagger}$, which implies that $a=a^*$, $b=c^*$, $c=b^*$, and $d=d*$. This means that $a$ and $d$ are real.

\vspace{1cm}

c)

\vspace{0.5cm}

Eigenvalues $\lambda$ for matrix $A$ satisfy $\abs{A-\lambda I}=0$.

\begin{equation*}
\abs{U-\lambda I}=\begin{vmatrix} a-\lambda & b \\ c & d-\lambda \end{vmatrix}
=(a-\lambda)(d-\lambda)-bc
\end{equation*}

\begin{equation*}
\abs{U^{\dagger}-\lambda^{\dagger} I}=\begin{vmatrix} a^*-\lambda^{\dagger} & c^* \\ b^* & d^*-\lambda^{\dagger} \end{vmatrix}
=(a^*-\lambda^{\dagger})(d^*-\lambda^{\dagger})-b^*c^*
\end{equation*}

Because $a=a^*$, $b=c^*$, $c=b^*$, and $d=d^*$, one sees that $\lambda$ must be equal to $\lambda^{\dagger}$, which implies that $\lambda$ is real. Here follows a more general proof that eigenvalues of a Hermitian matrix ($U=U^{\dagger}$) are real

\begin{align*}
U\vb{v} &= \lambda\vb{v} \\
(U\vb{v})^{\dagger} &= (\lambda\vb{v})^{\dagger} \\
\vb{v}^{\dagger} U^{\dagger} &= \lambda^{\dagger} \vb{v}^{\dagger} \\
\vb{v}^{\dagger} U^{\dagger} \vb{v} &= \lambda^{\dagger} \vb{v}^{\dagger} \vb{v} \\
(U&=U^{\dagger}) \\
\vb{v}^{\dagger} U \vb{v} &= \lambda^{\dagger} \vb{v}^{\dagger} \vb{v} \\
\vb{v}^{\dagger} \lambda \vb{v} &= \lambda^{\dagger} \vb{v}^{\dagger} \vb{v} \\
\lambda \vb{v}^{\dagger} \vb{v} &= \lambda^{\dagger} \vb{v}^{\dagger} \vb{v} \\
\lambda &= \lambda^{\dagger}
\end{align*}

Which must mean that the eigenvalues are real. A note on notation: the adjoint is to en operator what the complex conjugate is to numbers. $\lambda^{\dagger}$ is the complex conjugate of the eigenvalue.

\vspace{1cm}

d)

\vspace{0.5cm}

An operator $U$ is \emph{unitary} if $UU^{\dagger}=I$.

\begin{equation*}
UU^{\dagger}= 
\begin{pmatrix} a &  b \\  c & d \end{pmatrix} 
\begin{pmatrix} a^* & c^* \\ b^* & d^*\end{pmatrix}
= \begin{pmatrix} 
aa^*+bb^* & ac^* +bd^* \\
a^*c + b^*d & cc^* + dd^*
\end{pmatrix} 
= \begin{pmatrix} 
a^2+bc & ab + bd \\
ac + cd & bc + d^2
\end{pmatrix} 
\end{equation*}

This implies the following conditions:
\begin{itemize}
	\item $a^2+bc=1$ 
	\item $b(a+d)=0$
	\item $c(a+d)=0$
	\item $d^2+bc=1$
\end{itemize}

The trivial case is when $b=c=0$ and $a,d=\pm1$. If instead $a+d=0$ and $b,c\neq0$ there are other, more interesting solutions to the problem, for instance
\begin{equation*}
\begin{pmatrix}
\cos{\theta} & i\sine{\theta} \\
-i\sine{\theta} & -\cos{\theta}
\end{pmatrix},
\quad \theta \in [0,2\pi]
\end{equation*}

\vspace{1cm}

d)

\vspace{0.5cm}

The determinant of a unitary matrix must be 1, because $\det(U)\det(U^T)=\det(UU^T)=\det(1)=1$. Moreover,
\begin{equation*}
{\begin{vmatrix}
a & b \\ 
c & d
\end{vmatrix}}
= ad-bc = 1
\end{equation*}

then

\begin{equation*}
\abs{U-\lambda I} = \lambda^2-(a+d)\lambda+ad-bc = \lambda^2 -(a+d)\lambda+1=0
\end{equation*}

From above, $-(a+d)$ can have three possible values $-2$, $0$, $2$, which gives the possible eigenvalues of $\pm1$ and $\pm i$. Only real, eigenvalues are accepted if the matrix is Hermitian. If the matrix is both Hermitian and unitary, the matrix must therefore have unit egenvalues.

\end{document}