\documentclass{article}
\usepackage{amsmath}
\usepackage{physics}

\title{Problem Sheet 1 \\
\large{FYS3110}}
\author{Sebastian G. Winther-Larsen (sebastwi)}


\begin{document}

\maketitle

Problem 1.1

\hspace{1cm}

a)

\begin{align*}
\ket{\Psi} 		&= c(\sqrt{5}\ket{0}+i\ket{1})\\
\braket{\Psi} 	&= c^*(\sqrt{5}\bra{0}-i\bra{1}) c(\sqrt{5}\ket{0}+i\ket{1})\\
				&= c^*c(5\braket{0}+\sqrt{5}i\braket{0}{1}-\sqrt{5}\braket{1}{0}+\braket{1})\\
				&= \abs{c}(5+1) \rightarrow \abs{c}=\frac{1}{6}
\end{align*}

\hspace{1cm}

b)

The following representation is used for the basis vectors
\begin{align*}
\ket{0} &= \begin{pmatrix} 1 \\ 0 \end{pmatrix}\\
\ket{1} &= \begin{pmatrix} 0 \\ 1 \end{pmatrix}
\end{align*}

it follows that

\begin{align*}
\ket{\Psi} 	&= c(\sqrt{5}\ket{0}+i\ket{1}) \\
			&= c\left(
				\sqrt{5} \begin{pmatrix} 1 \\ 0 \end{pmatrix}
				i \begin{pmatrix} 0 \\ 1 \end{pmatrix}
				\right) \\
			&= c \begin{pmatrix} \sqrt{5} \\ i \end{pmatrix} \\[10pt]
\bra{\Psi} 	&= c^*(\sqrt{5}\bra{0}-i\bra{1}) \\
			&= c^*( \sqrt{5} ( 1 \quad 0 ) -i ( 0 \quad 1 )) \\
			&= c^* ( \sqrt{5} \quad  -i ) \\[10pt]
\braket{\Psi} &= \abs{c}6 \rightarrow \abs{c}=\frac{1}{6}
\end{align*}

Furthermore, $\hat{A}$ is an operator defined thusly


It is quite easy to see that 
\begin{equation*}

\end{equation*}

\end{document}