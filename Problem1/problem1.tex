\documentclass{article}
\usepackage{amsmath}
\usepackage{physics}

\title{Problem Sheet 1 \\
\large{FYS3110}}
\author{Sebastian G. Winther-Larsen (sebastwi)}


\begin{document}

\maketitle

\subsection*{Problem 1.1}

\hspace{1cm}

a)

\begin{align*}
\ket{\Psi} 		&= c(\sqrt{5}\ket{0}+i\ket{1})\\
\braket{\Psi} 	&= c^*(\sqrt{5}\bra{0}-i\bra{1}) c(\sqrt{5}\ket{0}+i\ket{1})\\
				&= c^*c(5\braket{0}+\sqrt{5}i\braket{0}{1}-\sqrt{5}\braket{1}{0}+\braket{1})\\
				&= \abs{c}(5+1) \rightarrow \abs{c}=\frac{1}{6}
\end{align*}

\hspace{1cm}

b)

\hspace{1cm}

The following representation is used for the basis vectors
\begin{align*}
\ket{0} &= \begin{pmatrix} 1 \\ 0 \end{pmatrix}\\
\ket{1} &= \begin{pmatrix} 0 \\ 1 \end{pmatrix}
\end{align*}

it follows that

\begin{align*}
\ket{\Psi} 	&= c(\sqrt{5}\ket{0}+i\ket{1}) \\
			&= c\left(
				\sqrt{5} \begin{pmatrix} 1 \\ 0 \end{pmatrix}
				i \begin{pmatrix} 0 \\ 1 \end{pmatrix}
				\right) \\
			&= c \begin{pmatrix} \sqrt{5} \\ i \end{pmatrix} \\[10pt]
\end{align*}

\begin{align*}
\bra{\Psi} 	&= c^*(\sqrt{5}\bra{0}-i\bra{1}) \\
			&= c^*( \sqrt{5} ( 1 \quad 0 ) -i ( 0 \quad 1 )) \\
			&= c^* ( \sqrt{5} \quad  -i ) \\[10pt]
\braket{\Psi} &= \abs{c}6 \rightarrow \abs{c}=\frac{1}{6}
\end{align*}

Furthermore, $\hat{A}$ is an operator defined thusly
\begin{equation*}
\hat{A}\ket{0} = -i\ket{1}, \quad \hat{A}\ket{1} = i\ket{0}
\end{equation*}


It is quite easy to see that 
\begin{equation*}
\hat{A} = \begin{pmatrix} 0 & i \\ -i & 0 \end{pmatrix}
\end{equation*}

$\hat{A}$ applied to $\ket{0}$:
\begin{equation*}
\hat{A}\ket{0} =
\begin{pmatrix} 0 & i \\ -i & 0 \end{pmatrix} \begin{pmatrix} 1 \\ 0 \end{pmatrix} = \begin{pmatrix} 0 \\ -i \end{pmatrix} = -i \begin{pmatrix} 0 \\ 1 \end{pmatrix} = -i\ket{1}
\end{equation*}


and $\hat{A}$ applied to $\ket{1}$:
\begin{equation*}
\hat{A}\ket{1} =
\begin{pmatrix} 0 & i \\ -i & 0 \end{pmatrix} \begin{pmatrix} 0 \\ 1 \end{pmatrix} = \begin{pmatrix} i \\ 0 \end{pmatrix} = i \begin{pmatrix} 1 \\ 0 \end{pmatrix} = i\ket{0}
\end{equation*}

\hspace{1cm}

c)

\hspace{1cm}

There are now two ways to calculate $\bra{\Psi}\hat{A}\ket{\Psi}$

\begin{align*}
\bra{\Psi}\hat{A}\ket{\Psi} &=
c^*\begin{pmatrix}\sqrt{5} & -i \end{pmatrix}
\begin{pmatrix} 0 & i \\ -i & 0 \end{pmatrix}
c\begin{pmatrix} \sqrt{5} \\ i \end{pmatrix} \\ &=
\abs{c} \begin{pmatrix} \sqrt{5} & -i \end{pmatrix}
\begin{pmatrix} -1 \\ -\sqrt{5}i \end{pmatrix} \\ &=
\frac{1}{6}(-\sqrt{5} - \sqrt{5}) = -\frac{\sqrt{5}}{3}
\end{align*}

\begin{align*}
\bra{\Psi}\hat{A}\ket{\Psi} &=
c^*(\sqrt{5}\bra{0}í\bra{1})\hat{A}c(\sqrt{5}\ket{0}+i\ket{1}) \\ &=
\abs{c}(\sqrt{5}\bra{0}-i\bra{1})(-\sqrt{5}i\ket{1}-\ket{0}) \\ &=
\frac{1}{6}(-5i\braket{0}{1}-\sqrt{5}\braket{0}{0}-\sqrt{5}\braket{1}{1}+i\braket{1}{0}) \\ &=
\frac{1}{6}(-2\sqrt{5}) = -\frac{\sqrt{5}}{3}
\end{align*}

\hspace{1cm}

\subsection*{Problem 1.2}

\hspace{1cm}

A complex matrix is given as

\begin{equation*}
U = \begin{pmatrix} a & b \\ c & d \end{pmatrix}
\end{equation*}

where $a, b, c, d$ are complex numbers.

\end{document}