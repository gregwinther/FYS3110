\documentclass{article}

\usepackage{physics}
\usepackage{amsmath}


\newcommand*\colvec[3][]{
    \begin{pmatrix}\ifx\relax#1\relax\else#1\\\fi#2\\#3\end{pmatrix}
}

\title{Problem Set II \\
  \large{FYS3110}}

\author{Sebastian G. Winther-Larsen (sebastwi)}
\date{\today}

\begin{document}

\maketitle

\section*{Problem 2.1}

An operator $\hat{H}$ is represented in a particular orthonormal basis
as the matrix
\begin{equation}
\hat{H}\simeq
\begin{pmatrix}
1 & \frac{i}{2} & 0 \\
-\frac{i}{2} & 1 & 0 \\
0 & 0 & \frac{1}{2}
\end{pmatrix}
\end{equation}

\subsection*{a)}

$\hat{H}$ is Hermitian if it is equal to its own transpose conjugate,
$\hat{H}=\hat{H}^{\dagger}$.

\begin{equation}
\hat{H}^{\dagger}=(\hat{H}^*)^T=
\begin{pmatrix}
1 & -\frac{i}{2} & 0 \\
\frac{i}{2} & 1 & 0 \\
0 & 0 & \frac{1}{2}
\end{pmatrix}^T=
\begin{pmatrix}
1 & \frac{i}{2} & 0 \\
-\frac{i}{2} & 1 & 0 \\
0 & 0 & \frac{1}{2}
\end{pmatrix}=
\hat{H}
\end{equation}

\subsection*{b)}

Three ket vectors are given 
\begin{align}
\ket{1} \simeq& \frac{1}{\sqrt{2}}\colvec[i]{1}{0}  \label{ket1}\\
\ket{2} \simeq& \colvec[0]{0}{1} \label{ket2}\\
\ket{3} \simeq& \frac{1}{\sqrt{3}}\colvec[-i]{1}{-1} \label{ket3}
\end{align}

$\ket{1}$ is and eigenvector of $\hat{H}$ with eigenvalue $\frac{3}{2}$:
\begin{equation}
\hat{H}\ket{1} = 
\begin{pmatrix}
1 & \frac{i}{2} & 0 \\
-\frac{i}{2} & 1 & 0 \\
0 & 0 & \frac{1}{2}
\end{pmatrix}
\frac{1}{\sqrt{2}}\colvec[i]{1}{0} =
\frac{1}{\sqrt{2}}\colvec[\frac{3i}{2}]{\frac{3}{2}}{0} =
\frac{3}{2}\ket{1}
\end{equation}

$\ket{2}$ is and eigenvector of $\hat{H}$ with eigenvalue $\frac{1}{2}$:
\begin{equation}
\hat{H}\ket{2} = 
\begin{pmatrix}
1 & \frac{i}{2} & 0 \\
-\frac{i}{2} & 1 & 0 \\
0 & 0 & \frac{1}{2}
\end{pmatrix}
\colvec[0]{0}{1} =
\colvec[0]{0}{\frac{1}{2}} =
\frac{1}{2}\ket{2}
\end{equation}

$\ket{3}$ is and eigenvector of $\hat{H}$ with eigenvalue $\frac{1}{2}$:
\begin{equation}
\hat{H}\ket{3} = 
\begin{pmatrix}
1 & \frac{i}{2} & 0 \\
-\frac{i}{2} & 1 & 0 \\
0 & 0 & \frac{1}{2}
\end{pmatrix}
\frac{1}{\sqrt{3}}\colvec[-i]{1}{-1} =
\frac{1}{\sqrt{3}}\colvec[-\frac{i}{2}]{\frac{1}{2}}{-\frac{1}{2}} =
\frac{1}{2}\ket{3}
\end{equation}

\subsection*{c)}

Computing the matrix elements of the linear operator

\begin{align}
\bra{1}\hat{H}\ket{1} &= \frac{3}{2}\braket{1} 
=\frac{3}{2}\frac{1}{2}\begin{pmatrix} i & 1 &
  0 \end{pmatrix}\colvec[i]{1}{0}
=\frac{3}{2}\frac{1}{2}2=\frac{3}{2}
\\
\bra{1}\hat{H}\ket{2} &= \frac{1}{2}\braket{1}{2} 
=\frac{1}{2}\frac{1}{\sqrt{2}}\begin{pmatrix} -i & 1 &
  0 \end{pmatrix}\colvec[0]{0}{1}
=0
\\
\bra{1}\hat{H}\ket{3} &= \frac{1}{2}\braket{1}{3} 
=\frac{1}{2}\frac{1}{\sqrt{2}}\frac{1}{\sqrt{3}}\begin{pmatrix} i & 1 &
  0 \end{pmatrix}\colvec[-i]{1}{-1}
= \frac{1}{2}\frac{1}{\sqrt{2}}\frac{1}{\sqrt{3}}2=\frac{1}{\sqrt{6}}
\end{align}

\begin{align}
\bra{2}\hat{H}\ket{1} &= \frac{3}{2}\braket{2}{1}
= \frac{3}{2}\frac{1}{\sqrt{2}}\begin{pmatrix} 0 & 0 &
  1 \end{pmatrix}\colvec[i]{1}{0} =0
\\
\bra{2}\hat{H}\ket{2} &= \frac{1}{2}\braket{2}{2} 
=\frac{1}{2}\begin{pmatrix} 0 & 0 & 1 \end{pmatrix}
\colvec[0]{0}{1}
=\frac{1}{2}
\\
\bra{2}\hat{H}\ket{3} &= \frac{1}{2}\braket{2}{3} 
=\frac{1}{2}\frac{1}{\sqrt{3}}\begin{pmatrix} 0 &  0 & 1 \end{pmatrix}
\colvec[-i]{1}{-1} = -\frac{1}{2\sqrt{3}}
\end{align}

\begin{align}
\bra{3}\hat{H}\ket{1} &=
\frac{3}{2}\braket{3}{1}=
\frac{3}{2}\frac{1}{\sqrt{2}}\frac{1}{\sqrt{3}}
\begin{pmatrix}
-i & 1 & -1
\end{pmatrix}
\colvec[i]{1}{0}
= \frac{3}{2}\frac{1}{\sqrt{2}}\frac{1}{\sqrt{3}}2
= \sqrt{\frac{3}{2}}
\\
\bra{3}\hat{H}\ket{2} &=
\\
\bra{3}\hat{H}\ket{3} &=
\end{align}



\end{document}