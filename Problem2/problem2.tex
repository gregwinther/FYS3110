\documentclass{article}

\usepackage[utf8]{inputenc}
\usepackage{physics}
\usepackage{amsmath}

\usepackage{epigraph}

\newcommand*\colvec[3][]{
    \begin{pmatrix}\ifx\relax#1\relax\else#1\\\fi#2\\#3\end{pmatrix}
}

\title{Problem Set II \\
  \hrulefill\fbox{\small{FYS3110}}\hrulefill}

\author{Sebastian G. Winther-Larsen}
\date{\today}

\begin{document}

\maketitle

%---------- Some lame ass quote -----------
\epigraph{\centering \large{\emph{``I invented the bra.''}}}{-- P.A.M. Dirac}


\section*{Problem 2.1}

An operator $\hat{H}$ is represented in a particular orthonormal basis
as the matrix
\begin{equation}
\hat{H}\simeq
\begin{pmatrix}
1 & \frac{i}{2} & 0 \\
-\frac{i}{2} & 1 & 0 \\
0 & 0 & \frac{1}{2}
\end{pmatrix}
\end{equation}

\subsection*{a)}

$\hat{H}$ is Hermitian if it is equal to its own transpose conjugate,
$\hat{H}=\hat{H}^{\dagger}$.

\begin{equation}
\hat{H}^{\dagger}=(\hat{H}^*)^T=
\begin{pmatrix}
1 & -\frac{i}{2} & 0 \\
\frac{i}{2} & 1 & 0 \\
0 & 0 & \frac{1}{2}
\end{pmatrix}^T=
\begin{pmatrix}
1 & \frac{i}{2} & 0 \\
-\frac{i}{2} & 1 & 0 \\
0 & 0 & \frac{1}{2}
\end{pmatrix}=
\hat{H}
\end{equation}

\subsection*{b)}

Three ket vectors are given 
\begin{align}
\ket{1} \simeq& \frac{1}{\sqrt{2}}\colvec[i]{1}{0}  \label{ket1}\\
\ket{2} \simeq& \colvec[0]{0}{1} \label{ket2}\\
\ket{3} \simeq& \frac{1}{\sqrt{3}}\colvec[-i]{1}{-1} \label{ket3}
\end{align}

$\ket{1}$ is and eigenvector of $\hat{H}$ with eigenvalue $\frac{3}{2}$:
\begin{equation}
\hat{H}\ket{1} = 
\begin{pmatrix}
1 & \frac{i}{2} & 0 \\
-\frac{i}{2} & 1 & 0 \\
0 & 0 & \frac{1}{2}
\end{pmatrix}
\frac{1}{\sqrt{2}}\colvec[i]{1}{0} =
\frac{1}{\sqrt{2}}\colvec[\frac{3i}{2}]{\frac{3}{2}}{0} =
\frac{3}{2}\ket{1}
\end{equation}

$\ket{2}$ is and eigenvector of $\hat{H}$ with eigenvalue $\frac{1}{2}$:
\begin{equation}
\hat{H}\ket{2} = 
\begin{pmatrix}
1 & \frac{i}{2} & 0 \\
-\frac{i}{2} & 1 & 0 \\
0 & 0 & \frac{1}{2}
\end{pmatrix}
\colvec[0]{0}{1} =
\colvec[0]{0}{\frac{1}{2}} =
\frac{1}{2}\ket{2}
\end{equation}

$\ket{3}$ is and eigenvector of $\hat{H}$ with eigenvalue $\frac{1}{2}$:
\begin{equation}
\hat{H}\ket{3} = 
\begin{pmatrix}
1 & \frac{i}{2} & 0 \\
-\frac{i}{2} & 1 & 0 \\
0 & 0 & \frac{1}{2}
\end{pmatrix}
\frac{1}{\sqrt{3}}\colvec[-i]{1}{-1} =
\frac{1}{\sqrt{3}}\colvec[-\frac{i}{2}]{\frac{1}{2}}{-\frac{1}{2}} =
\frac{1}{2}\ket{3}
\end{equation}

The last two eigenvalues $\lambda_2 = \lambda_3 = \frac{1}{2}$ are degenerate.

\subsection*{c)}

Computing the matrix elements of the linear operator

\begin{align}
\bra{1}\hat{H}\ket{1} &= \frac{3}{2}\braket{1} 
=\frac{3}{2}\frac{1}{2}\begin{pmatrix} -i & 1 &
  0 \end{pmatrix}\colvec[i]{1}{0}
=\frac{3}{2}\frac{1}{2}2=\frac{3}{2}
\\
\bra{1}\hat{H}\ket{2} &= \frac{1}{2}\braket{1}{2} 
=\frac{1}{2}\frac{1}{\sqrt{2}}\begin{pmatrix} -i & 1 &
  0 \end{pmatrix}\colvec[0]{0}{1}
=0
\\
\bra{1}\hat{H}\ket{3} &= \frac{1}{2}\braket{1}{3} 
=\frac{1}{2}\frac{1}{\sqrt{2}}\frac{1}{\sqrt{3}}\begin{pmatrix} i & 1 &
  0 \end{pmatrix}\colvec[-i]{1}{-1}
= \frac{1}{2}\frac{1}{\sqrt{2}}\frac{1}{\sqrt{3}}0=0
\end{align}

\begin{align}
\bra{2}\hat{H}\ket{1} &= \frac{3}{2}\braket{2}{1}
= \frac{3}{2}\frac{1}{\sqrt{2}}\begin{pmatrix} 0 & 0 &
  1 \end{pmatrix}\colvec[i]{1}{0} =0
\\
\bra{2}\hat{H}\ket{2} &= \frac{1}{2}\braket{2}{2} 
=\frac{1}{2}\begin{pmatrix} 0 & 0 & 1 \end{pmatrix}
\colvec[0]{0}{1}
=\frac{1}{2}
\\
\bra{2}\hat{H}\ket{3} &= \frac{1}{2}\braket{2}{3} 
=\frac{1}{2}\frac{1}{\sqrt{3}}\begin{pmatrix} 0 &  0 & 1 \end{pmatrix}
\colvec[-i]{1}{-1} = -\frac{1}{2\sqrt{3}}
\end{align}

\begin{align}
\bra{3}\hat{H}\ket{1} &=
\frac{3}{2}\braket{3}{1}=
\frac{3}{2}\frac{1}{\sqrt{2}}\frac{1}{\sqrt{3}}
\begin{pmatrix}
i & 1 & -1
\end{pmatrix}
\colvec[i]{1}{0}
= \frac{3}{2}\frac{1}{\sqrt{2}}\frac{1}{\sqrt{3}}0
= 0
\\
\bra{3}\hat{H}\ket{2} &=
\frac{1}{2}\braket{3}{2}=
\frac{1}{2}\frac{1}{\sqrt{3}}
\begin{pmatrix}
i & 1 & -1
\end{pmatrix}
\colvec[0]{0}{1}=
\frac{1}{2}\frac{1}{\sqrt{3}}(-1)=-\frac{1}{2\sqrt{3}}
\\
\bra{3}\hat{H}\ket{3} &=
\frac{1}{2}\braket{3}=
\frac{1}{2}\frac{1}{3}
\begin{pmatrix}
i & 1 & -1
\end{pmatrix}
\colvec[-i]{1}{-1}=
\frac{1}{2}\frac{1}{3}3=\frac{1}{2}
\end{align}

Thus, the full matrix is
\begin{equation}
H_{ij}=\begin{pmatrix}
\frac{3}{2} & 0 & 0 \\
0 & \frac{1}{2} & -\frac{1}{2\sqrt{3}} \\
0 & -\frac{1}{2\sqrt{3}} & \frac{1}{2}
\end{pmatrix}
\end{equation}

Evidently, the matrix $H_{ij}$ is not diagonal, but is symmetric with the eigenvalues as main diagonal entries.

\subsection*{d)}

From the results when $H_{ij}$ was computed one can already gather that $\ket{2}$ and $\ket{3}$ are non-orthogonal. The Gram-Schmidt precedure can be employed to convert a linearly independent basis into an orthonormal one, but it is not necessary to do all the work. It is easy to see that the kets are already of unit length as $\sqrt{\braket{i}{i}} = 1, i=1,2,3$. A new orthonomral basis is computed as follows.

Everything is already fine with the first ket:

\begin{equation}
\ket{1'}=\ket{1}
\end{equation}

The second vector in the new basis is the $\ket{2}$ minus the part pointing along the first vector. 

\begin{equation}
\ket{2'}=\ket{2}-\ket{1'}\braket{1'}{2}=\ket{2}
\end{equation}

The second ket in the new basis is also equal to its ``predecessor'', because no part of it points along the first ket ($\braket{1'}{2}=0$).

Finally, the third new basis vector is computed by subtracting both the part of it that points along the first and the second vector.

\begin{align}
\ket{\tilde{3}}&=\ket{3}-\ket{1'}\braket{1'}{3}-\ket{2'}\braket{2'}{3}
=\ket{3}-\ket{2'}\braket{2'}{3}\\
&=\frac{1}{\sqrt{3}}\colvec[-i]{1}{-1} - 
\frac{1}{\sqrt{3}}\colvec[0]{0}{1}\begin{pmatrix}0 & 0 & 1\end{pmatrix}\colvec[-i]{1}{-1}=\frac{1}{\sqrt{3}}\colvec[-i]{1}{0}
\end{align}

Now this vector must be normalised

\begin{equation}
\ket{3'}=\frac{\ket{\tilde{3}}}{\sqrt{\braket{\tilde{3}}}}
=\frac{\ket{\tilde{3}}}{\sqrt{\frac{1}{3}\begin{pmatrix}i & 1 & 0\end{pmatrix}\colvec[-1]{1}{0}}}=
\frac{\ket{\tilde{3}}}{\sqrt{\frac{2}{3}}}=\frac{1}{\sqrt{2}}\colvec[-i]{1}{0}
\end{equation}

One should have been able to see that one coming from far away. It is easy to see that 
\begin{equation}
\braket{i'}{j'}=
\begin{cases}
1, & i=j\\
0, & i\neq j
\end{cases}
\end{equation}
Therefore, $H_{i'j'}$ must be a diagonal matrix. The only unknown element is
\begin{equation}
\bra{3'}\hat{H}\ket{3'}=
\frac{1}{2}\begin{pmatrix}
i & 1 & 0
\end{pmatrix}\begin{pmatrix}
1 & \frac{i}{2} & 0 \\
\frac{i}{2} & 1 & 0 \\
0 & 0 & \frac{1}{2}
\end{pmatrix}
\colvec[-i]{1}{0}=
\frac{1}{2}\begin{pmatrix}
i & 1 & 0
\end{pmatrix}
\colvec[-\frac{i}{2}]{\frac{1}{2}}{0}=\frac{1}{2}
\end{equation}
Not surprisingly, the last eigenvalue is the same and the new matrix is
\begin{equation}
H_{i'j'}=\begin{pmatrix}
\frac{3}{2} & 0 & 0 \\
0 & \frac{1}{2} & 0 \\
0 & 0 & \frac{1}{2}
\end{pmatrix}
\end{equation}

\section*{Problem 2.2}
\subsection*{a)}
Here I will determine the Hermitian conugate of operators $i$, $x^2$ and $\frac{d}{dx}$. Given an operator $\hat{O}$, its Hermitian conjugate $\hat{O}^{\dagger}$ satisfies the following property by definition.
\begin{equation}
\label{hermconj}
\braket{\phi}{\hat{O}\psi} = \braket{\hat{O}^{\dagger}\phi}{\psi} 
\end{equation} 
where $\bra{\phi}$ and $\ket{\psi}$ are simply an arbitrary bra and ket, respectively. Given a (complex) number $\alpha$, the following properties are important.
\begin{align}
\alpha\ket{\psi} &= \ket{\psi} \\
\bra{\alpha\phi} &= \bra{\phi}\alpha^*
\end{align}
Therefore, in general an operator which is simply an complex number will have its complex conjugate as Hermitian conjugate.
\begin{equation}
\braket{(a-ib)\phi}{\psi}=\braket{\phi}{(a+ib)\psi}
\end{equation}

Bearing this result in mind, the Hermitian conjugates of $i$ and $x^2$ are simply their complex conjugates, $(i)^*=-i$ and $(x^2)^*=x^2$. Now, to compute the Hermitian conjugate of $\frac{d}{dx}$ I will derive the result using integration by parts.
\begin{align}
\braket{\phi}{\frac{d}{dx}\psi}&=\int_{-\infty}^{\infty}\phi^*(x)\frac{d\psi(x)}{dx} dx\\
&=\left[\phi^*(x)\psi(x)\right]_{-\infty}^{\infty}
-\int_{-\infty}^{\infty}\frac{d\phi^*(x)}{x}\psi(x) dx \label{intparts}\\
&= \braket{-\frac{d}{dx}\phi}{\psi}
\end{align}
The first term in (\ref{intparts}) is zero, consequently providing the result
\begin{equation}
\left(\frac{d}{dx} \right)^{\dagger} = -\frac{d}{dx}
\end{equation}

\subsection*{b)}
By employing the property in (\ref{hermconj}) it is quite straightforward to compute the Hermitian conjugate of a composit operator $\hat{O}=\hat{K}\hat{L}$
\begin{equation}
\braket{\phi}{\hat{K}\hat{L}\psi}=\braket{\hat{K}^{\dagger}\phi}{\hat{L}\psi}=\braket{\hat{L}^{\dagger}\hat{K}^{\dagger}\phi}{\psi}
\end{equation}
or, somewhat more condensely written
\begin{equation}
(\hat{K}\hat{L})^{\dagger}=\hat{L}^{\dagger}\hat{K}^{\dagger}
\end{equation}

\subsection*{c)}

The vector $\ket{\lambda}$ is an eigenket og the Hermitian operator $\hat{K}$ with eigenvalue $\lambda$, such that $\hat{K}\ket{\lambda}=\lambda\ket{\lambda}$. Now the task is to prove that
\begin{equation}
\bra{\lambda}\hat{K}\hat{L}\ket{g}=\bra{\lambda}\hat{L}\ket{g}\lambda
\end{equation}

Firstly, because of the properties of Hermitian and composite operators
\begin{equation}
(\hat{K}\hat{L})^{\dagger}=\hat{L}^{\dagger}\hat{K}^{\dagger}=\hat{L}\hat{K}
\end{equation}

Now for the proof
\begin{equation}
\bra{\lambda}\hat{K}\hat{L}\ket{g}=\bra{\lambda}\hat{L}\hat{K}\ket{g}=
(\bra{g}\hat{L}\hat{K}\ket{\lambda})^*=(\bra{g}\hat{L}\lambda\ket{\lambda})^*=\bra{\lambda}\hat{L}\ket{g}\lambda
\end{equation}

\section*{2.3}

The following system is under scrutiny. A Hamiltonian $\hat{H}$ is linear and acts as follows
\begin{equation*}
\hat{H}\ket{\psi}=g\ket{\phi}, \quad \hat{H}\ket{\phi}=g^*\ket{\psi}, \quad \hat{H}\ket{\gamma_n}=0
\end{equation*}
where $g$ is and arbitrary complex number, $\ket{\psi}$ and $\ket{\phi}$ is a pair of linearly independent states, normalised to unity\footnote{Not necesarily orthogonal}, and $\ket{\gamma_n}$, $n\in[1,N]$ are all sates orthogonal to both $\ket{\psi}$ and $\ket{\phi}$.

From the two problems above one can conlude that computing $\bra{\lambda}\hat{O}\ket{\lambda}$ will produce the eigenvalues of eigenstate $\ket{\lambda}$ for unit length vectors. The eigenvalues were the same whether the states were orthogonal or not, it is enough for them to be linearly independent, as is the case here, because then they are a basis for the system. That means that they are also eigenstates for the system.
\begin{align}
\bra{\psi}\hat{H}\ket{\psi}&=\braket{\psi}{\hat{H}\psi}=\braket{\psi}{g\phi}=g\braket{\psi}{\phi} \\
&=\braket{\hat{H}\psi}{\psi}=\braket{g\psi}{\psi}=g^*\braket{\phi}{\psi}=g^*\braket{\psi}{\phi}=g^*\braket{\phi}{\psi}
\end{align} 

Which gives the eigenvalue for eigenstate $\ket{\psi}$ as $g\braket{\phi}{\psi}=g^*\braket{\psi}{\phi}^*$. Similarly, the eigenvalue for eigenstate $\ket{\phi}$ is $g\braket{\psi}{\phi}=g^*\braket{\phi}{\psi}^*$. Because $\ket{\gamma_n}$ are orthogonal to both $\ket{\psi}$ and $\ket{\phi}$, they must also be eigenstates, but are not as interesting because their eigenvalues are zero.

Additionally, one can prove that all eigenvalues of an Hermitian operator are real. Consider
\begin{align}
\hat{O}\ket{\lambda} &= \lambda\ket{\lambda} \\
\bra{\lambda}\hat{O}\ket{\lambda} &= \lambda\braket{\lambda}{\lambda} \label{this}\\
\bra{\lambda}\hat{O}^{\dagger}\ket{\lambda} &= \lambda^*\braket{\lambda}{\lambda} \\
\bra{\lambda}\hat{O}\ket{\lambda} &= \lambda^*\braket{\lambda}{\lambda} \label{that}
\end{align}
subtracting \ref{this} from \ref{that} yields
\begin{align}
0=(\lambda&-\lambda^*)\braket{\lambda}\\
\lambda&=\lambda^*
\end{align}
which only holds for real numbers.

\end{document}

