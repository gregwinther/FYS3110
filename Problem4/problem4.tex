\documentclass{article}

\usepackage{amsmath}
\usepackage{physics}

\title{Problem Sheet 4 \\
\large{FYS3110}}

\author{Sebastian G. Winther-Larsen}

\date{\today}

\begin{document}

\maketitle

The hamiltonian of particle with mass $m$ in a one-dimensional oscillator potetial having a characteristic frequency $\omega$ is
\begin{equation}
\hat{H}=\frac{1}{2m}\hat{p}^2+\frac{1}{2}m\omega^2\hat{X}^2
\end{equation}
The ladder operators for the harmonic oscillator potential are
\begin{align}
\hat{a}&=\frac{1}{\sqrt{2\hbar m\omega}}(m\omega\hat{X}+i\hat{P})
&\text{(lowering operator)} \label{eq:ladderdown} \\
\hat{a}^{\dagger}&=\frac{1}{\sqrt{2\hbar m\omega}}(m\omega\hat{X}-i\hat{P})
&\text{(raising operator)} \label{eq:ladderup}
\end{align}

\section*{Problem 4.1}

\subsection*{a)}

I want to find an expression for $\hat{X}$ in terms of $\hat{a}_{nm}$ and $\hat{a}^{\dagger}_{nm}$. This can be done by first rewriting equation \ref{eq:ladderdown}
\begin{align*}
\hat{a}=\frac{1}{\sqrt{2\hbar m\omega}}(m\omega\hat{X}+i\hat{P})
= \sqrt{\frac{m\omega}{2\hbar}}\hat{X}+\frac{i}{\sqrt{2\hbar m \omega}}\hat{P} \\
\rightarrow \hat{X} = \sqrt{\frac{2\hbar}{m\omega}}\hat{a}-\sqrt{\frac{2\hbar}{m\omega}}\frac{i}{\sqrt{2\hbar m\omega}}\hat{P}, 
\end{align*}
and then equation \ref{eq:ladderup}
\begin{align*}
\hat{a}^{\dagger} = \frac{1}{\sqrt{2\hbar m\omega}}(m\omega\hat{X}-i\hat{P}) = \sqrt{\frac{2\hbar}{m\omega}}\hat{X}+\frac{i}{2\hbar m\omega}\hat{P} \\
\rightarrow \hat{P}= \frac{2\hbar m \omega}{i}\sqrt{\frac{m\omega}{2\hbar}}\hat{X}-\frac{2\hbar m\omega}{i}\hat{a}^{\dagger}.
\end{align*}
Now putting the latter equation into the former yields
\begin{align*}
\hat{X}&=\sqrt{\frac{2\hbar}{m\omega}}\hat{a}-\sqrt{\frac{2\hbar}{m\omega}}\frac{i}{2\hbar m\omega}\left(\frac{2\hbar m\omega}{i}\sqrt{\frac{m\omega}{2\hbar}}\hat{X}-\frac{2\hbar m\omega}{i}\hat{a} \right) \\
&=\sqrt{\frac{2\hbar}{m\omega}}\hat{a}+\sqrt{\frac{2\hbar}{m\omega}}\hat{a}^{\dagger}-\sqrt{\frac{2\hbar}{m\omega}}\sqrt{\frac{m\omega}{2\hbar}}\hat{x},
\end{align*}
which simplifies to
\begin{equation}
\label{eq:xoperator}
\hat{X}=\sqrt{\frac{\hbar}{2m\omega}}(\hat{a}+\hat{a}^{\dagger})
\end{equation}

It will be necessary to know what $[\hat{a},\hat{H}]$ is. This is easiest to compute if one knows how $\hat{a}$ and $\hat{a}^{\dagger}$ is related to $\hat{H}$. By looking at the expressions for $\hat{a}$ and $\hat{a}^{\dagger}$ one is tempted to compute the following
\begin{equation*}
\hat{a}\hat{a}^{\dagger}=\frac{m\omega}{2\hbar}\hat{X}^2+\frac{1}{2m\omega\hbar}\hat{P}^2+\frac{i}{2\hbar}[\hat{X},\hat{P}],
\end{equation*}
where $[X,P]=i\hbar$, which follows from $\hat{X}\rightarrow x$ and $\hat{P}\rightarrow i\hbar (d/dx)$, but is independent of basis. So we see that
\begin{equation}
\label{eq:hamilton}
\hat{H} = \hbar\omega(\hat{a}\hat{a}^{\dagger}+\frac{1}{2}),
\end{equation}
or alternatively that $\hat{a}\hat{a}^{\dagger}=\frac{\hat{H}}{\hbar \omega}-\frac{1}{2}$.
This combined ladder operator can be referred to by a new name $\hat{a}\hat{a}^{\dagger}=\hat{N}$, so that equation \ref{eq:hamilton} becomes
\begin{equation}
\hat{H} = \hbar\omega(\hat{N}+\frac{1}{2}),
\end{equation}
Thus, we have the energy eigenbasis which satisfy
\begin{equation}
\hat{H}\ket{n}=\hbar\omega(n-\frac{1}{2})\ket{n}, \text{ for } i \in 0, 1, 2 \dots
\end{equation}

We now get
\begin{equation}
\hat{a}\ket{n} = C_n\ket{n},
\end{equation}
where $C_n$ is a constant which can be found the following way
\begin{align*}
\bra{n}\hat{a}^{\dagger}\hat{a}\ket{n} = & \abs{C_n}^2\braket{n-1}\\
\rightarrow & \abs{C_n}=\sqrt{n}=C_n,
\end{align*}
by choosing the phase to be zero\footnote{Actually, $C_n=\sqrt{n}e^{i\phi}$ where $\phi$ is arbitrary, but is is conventional to set $\phi=0$.}. We land at
\begin{equation}
\hat{a}\ket{n}=\sqrt{n}\ket{n-1}
\end{equation}
Similarly for $\hat{a}^{\dagger}$
\begin{equation}
\hat{a}^{\dagger}\ket{n} = D_n\ket{n+1}
\end{equation}
\begin{align*}
\bra{n}\hat{a}\hat{a}^{\dagger}\ket{n}=&\abs{D_n}^2\braket{n+1} \\
\rightarrow & \abs{D_n}=\sqrt{n+1}=D_n
\end{align*}
\begin{equation}
\hat{a}^{\dagger}\ket{n}=\sqrt{n+1}\ket{n+1}
\end{equation}

Now to compute the matrix elements for $\hat{a}$ and $\hat{a}^{\dagger}$
\begin{align}
\bra{m}\hat{a}\ket{n}=\sqrt{n}\braket{m}{n-1}=&\sqrt{n}\delta_{m,n-1} \\
\bra{m}\hat{a}^{\dagger}\ket{n}=\sqrt{n-1}\braket{m}{n+1}=&\sqrt{n-1}\delta_{m,n+1}.
\end{align}
The actual matrices for these particular operators will look something like this\footnote{I am perfectly aware that $\sqrt{1}=1$ but I will keep the root symbol here in order to underline the symmetry.}
\begin{align}
a &\leftrightarrow \begin{bmatrix}
0 & \sqrt{1} & 0 & 0 & \dots \\
0 & 0 & \sqrt{2} & 0 &  \\
0 & 0 & 0 & \sqrt{3} & \\
\vdots & & & & \ddots 
\end{bmatrix} \\
a^{\dagger} &\leftrightarrow \begin{bmatrix}
0 & 0 & 0 & \dots \\
\sqrt{1} & 0 & 0 & \\
0 & \sqrt{2} & 0 & \\
0 & 0 & \sqrt{3} & \\
\vdots & & & \ddots 
\end{bmatrix}.
\end{align}

Finding the matrix representation of $\hat{X}$ is now an easy matter of employing equation \ref{eq:xoperator}
\begin{equation}
X \leftrightarrow  \begin{bmatrix}
0 & \sqrt{1} & 0 & 0 & \dots \\
\sqrt{1} & 0 & \sqrt{2} & 0 &  \\
0 & \sqrt{2} & 0 & \sqrt{3} & \\
0 & 0 & \sqrt{3} & 0 \\
\vdots & & & & \ddots 
\end{bmatrix},
\end{equation}
while an algebraic expression will  be (also employing equation \ref{eq:xoperator})
\begin{align}
\bra{m}X\ket{n} &= \sqrt{\frac{\hbar}{2m\omega}}(\sqrt{n+1}\braket{m}{n+1}+\sqrt{n}\braket{m}{n-1}) \\
&= \sqrt{\frac{\hbar}{2m\omega}}(\sqrt{n+1}\delta_{m,n+1}-\sqrt{n}\delta_{m,n-1}).
\end{align}

\subsection*{b)}
Let $\ket{\psi(0)}=\sum_{x=0}^n c_n\ket{n}$. It follows that
\begin{align*}
\braket{\psi(0)}=
\end{align*}

\end{document}