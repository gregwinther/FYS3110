\documentclass[]{article}

\usepackage{physics}
\usepackage{amsmath}
\usepackage{bbold}
\usepackage[export]{adjustbox}

\usepackage[utf8]{inputenc}

\title{Problem Sheet 6 \\
	\large{FYS3110}}
	
\author{Sebastian G. Winther-Larsen}

\date{\today}

\begin{document}

\maketitle


\begin{figure}[!htbp]
	\begin{minipage}[b]{0.5\textwidth}
	\adjincludegraphics[width=\textwidth, trim={0 {0.51\height} 0 0}, clip]{figures/futuramameasurement.jpg}
	\end{minipage}
	\begin{minipage}[b]{0.5\textwidth}
	\adjincludegraphics[width=\textwidth, trim={0 0 0 {0.51\height}}, clip]{figures/futuramameasurement.jpg}
	\end{minipage}
	\caption{Quantum joke from Futurama.}
\end{figure}

\section*{Problem 6.1}s

An electron which has spin-$1/2$ is in the states
\begin{equation}
\label{eq:thestate}
\ket{\psi} = \sqrt{\frac{2}{5}}\ket{3,2,1}\otimes\ket{\downarrow_z}+\sqrt{\frac{3}{5}}\ket{2,1,1}\otimes\ket{\uparrow_z},
\end{equation}
of the hydrogen atom. The state with quantum numbers $n,l,m$ and spin $s_z=\{\uparrow_z=\hbar/2, \quad \downarrow_z=-\hbar/2\}$ along the $z$-axis is denoted $\ket{n,l,m}\otimes\ket{s_z}$.

\subsection*{a)}
The probability that the electron is measured to be in the spin up state along the $z$-axis can be calculated in a very difficult manner by computing $\bra{\psi}(\mathbb{1}\otimes\ket{\uparrow_z}\bra{\uparrow_z}) \ket{\psi}$. However, one can simply look at the coefficients in the superposition represesenation of the state in equation \ref{eq:thestate} and realize that 
\begin{equation}
\label{eq:probupanddown}
P(\uparrow_z)=\frac{3}{5}, \quad P(\downarrow_z)=\frac{2}{5}.
\end{equation}
The probabilities add up to one, implying that the state is normalized.

\subsection*{b)}

To find which values can be measured for $L^2$ and for what probabilities one need simply to apply the $\hat{L}^2$ operator to the state of the electron in equation \ref{eq:thestate}.
\begin{align*}
\hat{L}^2\ket{\psi}
&=\sqrt{\frac{2}{5}}\hat{L}^2\ket{3,2,1}\otimes\ket{\downarrow_z}
+ \sqrt{\frac{3}{5}}\hat{L}^2\ket{2,1,1}\otimes\ket{\uparrow_z} \\
&=\sqrt{\frac{2}{5}}(\hbar^22(2+1)\ket{3,2,1}\otimes\ket{\downarrow_z})
+ \sqrt{\frac{3}{5}}(\hbar^21(1+1)\ket{2,1,1}\otimes\ket{\uparrow_z}) \\
&=\sqrt{\frac{2}{5}}(6\hbar\ket{3,2,1}\otimes\ket{\downarrow_z})
+ \sqrt{\frac{2}{5}}(2\hbar^2\ket{2,1,1}\otimes\ket{\uparrow_z})
\end{align*}
which means that one measures $6\hbar^2$ with probability $2/5$ and $2\hbar^2$ with probability $3/5$ for $L^2$. 

The corresponding computation for $L_z$ is
\begin{align*}
\hat{L}_z\ket{\psi}
&=\sqrt{\frac{2}{5}}(\hat{L}_z\ket{3,2,1}\otimes\ket{\downarrow_z})
+ \sqrt{\frac{3}{5}}(\hat{L}_z\ket{2,1,1}\otimes\ket{\uparrow_z}) \\
&=\sqrt{\frac{2}{5}}(\hbar\ket{3,2,1}\otimes\ket{\downarrow_z})
+ \sqrt{\frac{3}{5}}(\hbar\ket{2,1,1}\otimes\ket{\uparrow_z}) \\
\end{align*}
which means that one measures $\hbar$ with probability $1$ for $L_z$.

Lastly, the computation for $S^2$
\begin{align*}
\hat{L}_z\ket{\psi} 
&=\sqrt{\frac{2}{5}}\left(\hbar^2\frac{1}{2}\frac{3}{2}\ket{3,2,1}\otimes\ket{\downarrow_z}\right)
+ \sqrt{\frac{3}{5}}\left(\hbar^2\frac{1}{2}\frac{3}{2}\ket{2,1,1}\otimes\ket{\uparrow_z}\right)\\
&=\sqrt{\frac{2}{5}}\left(\frac{3\hbar^2}{4}\ket{3,2,1}\otimes\ket{\downarrow_z}\right)
+ \sqrt{\frac{3}{5}}\left(\frac{3\hbar^2}{4}\ket{3,2,1}\otimes\ket{\uparrow_z}\right)
\end{align*}
which means that one measure $\frac{3\hbar^2}{4}$ with probability $1$ for $S^2$.

\subsection*{c)}
Now, consider the total angular momentum $\vb{J}=\vb{L}+\vb{S}$ (really $\vb{L}\otimes I + I \otimes\vb{S}$).

First considering the superposed state from equation \ref{eq:thestate}
\begin{align*}
\ket{\psi}
&=\sqrt{\frac{2}{5}}\ket{3,2,1}\otimes\ket{\downarrow_z}
+ \sqrt{\frac{3}{5}}\ket{2,1,1}\otimes\ket{\uparrow_z} \\
&=\sqrt{\frac{2}{5}}R_{3,2}Y_2^1\ket{\frac{1}{2},-\frac{1}{2}}
+ \sqrt{\frac{3}{5}}R_{2,1}Y_1^1\ket{\frac{1}{2},\frac{1}{2}}
\end{align*}
Because the radial wave function is unnecessary to calculate total angular momentum this can be simplified to
\begin{equation}
\label{eq:simplepsi}
\ket{\psi}
= \sqrt{\frac{2}{5}}\ket{\frac{1}{2},-\frac{1}{2}}\ket{2,1} 
+ \sqrt{\frac{3}{2}}\ket{\frac{1}{2},\frac{1}{2}}\ket{1,1}
\end{equation}
Employting Glebsch-Gordan coefficient tables one can find that
\begin{align}
\label{eq:glebschgordan1}
\ket{\frac{1}{2},-\frac{1}{2}}\ket{2,1} &=\sqrt{\frac{2}{5}}\ket{\frac{5}{2},\frac{1}{2}}+ \sqrt{\frac{3}{5}}\ket{\frac{3}{5},\frac{1}{2}} \\
\label{eq:glebschgordan2}
\ket{\frac{1}{2},\frac{1}{2}}\ket{1,1} &= \ket{\frac{1}{2},\frac{1}{2}}
\end{align}
Inserting \ref{eq:glebschgordan1} and \ref{eq:glebschgordan2} into \ref{eq:simplepsi} yields
\begin{align*}
\ket{\psi} = \sqrt{\frac{2}{5}}\left(\sqrt{\frac{2}{5}}\ket{\frac{5}{2},\frac{1}{2}} + \sqrt{\frac{3}{5}}\ket{\frac{3}{2},\frac{1}{2}}\right) + \sqrt{\frac{3}{2}}\ket{\frac{1}{2},\frac{1}{2}} \\ = \sqrt{\frac{4}{25}}\ket{\frac{5}{2},\frac{1}{2}} + \sqrt{\frac{15}{25}}\ket{\frac{3}{2},\frac{1}{2}}+\sqrt{\frac{15}{25}}\ket{\frac{1}{2},\frac{1}{2}}
\end{align*}
One can clearly see that the superposition is normalized, which is good and expected.
\end{document}