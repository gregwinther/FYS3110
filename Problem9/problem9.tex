\documentclass[11pt]{amsart}

\usepackage{amsmath}
\usepackage{physics}

\title[Problem Sheet 9]{Problem Sheet 9\\
		\large{FYS3110}}
		
\author[Winther-Larsen]{Sebastian G. Winther-Larsen}

\begin{document}

\maketitle

\section*{Problem 9.1}

For the harmonic oscillator the potential is $V(x) = \frac{1}{2}kx^2$ and the allowed energies are
\begin{equation}
E_n = \left(n + \frac{1}{2} \right)\hbar \omega, \text{ for } n = 0,1,2,\dots
\end{equation}
where $\omega = \sqrt{\frac{k}{m}}$ is the classical angular frequency. 

\subsection*{a}
The spring constant is increased slightly from $k$ to $(1+\epsilon)k$. The exact new allowed energies are
\begin{equation}
\label{eq:harmonicenergies}
E_n = \left(n + \frac{1}{2} \right)\hbar\sqrt{\frac{(1+\epsilon)k}{m}}.
\end{equation}
The MacLaurin series\footnote{Taylor expansion around zero, from which the power series arises.} of the increased spring constant up to second order is
\begin{equation}
\label{eq:powerseries}
\sqrt{1+\epsilon} \approx 1 + \frac{\epsilon}{2} - \frac{\epsilon^2}{8} \dots
\end{equation}
Inserting equation \ref{eq:powerseries} into \ref{eq:harmonicenergies} yields
\begin{equation}
E_n \approx \left(n + \frac{1}{2} \right)\hbar\sqrt{\frac{k}{m}}\left(1 + \frac{\epsilon}{2} - \frac{\epsilon}{8}\right) 
\end{equation}

\subsection*{b}
Now to calculate the first-order perturbation in the energy
\begin{equation}
\label{eq:firstorderpeturbationenergy}
E_n^1 = \bra{\psi_n^0}H'\ket{\psi_n^0},
\end{equation}
where $H' = T + V'$ and $V' = \frac{1+\epsilon}{2}kx^2$. The change in change in energy is
\begin{equation*}
H' - H = V' - V = \frac{1+\epsilon}{2}kx^2 - \frac{1}{2}kx^2 = \frac{1}{2}\epsilon kx^2 = \epsilon V, 
\end{equation*}
which reduces equation \ref{eq:firstorderpeturbationenergy} to
\begin{equation}
\label{eq:firstorderpeturbationenergy1}
E_n^1 = \bra{\psi_n^0}\epsilon V \ket{\psi_n^0}.
\end{equation}

This equation can be solved quite easily be employing the virial theorem for a stationary state
\begin{equation}
\label{eq:virial}
2\ev{T} = \ev{x\frac{dV}{dx}}.
\end{equation}
For the harmonic oscillator
\begin{equation*}
\ev{x\frac{dV}{dx}}= k \ev{x^2} \rightarrow \ev{T} =  k \ev{x^2} \rightarrow \ev{T} = \frac{1}{2}k\ev{x^2} = \ev{V} = \frac{E_n}{2}.
\end{equation*}
It follows that equation \ref{eq:firstorderpeturbationenergy1} becomes
\begin{equation}
E_n^1 = \frac{\epsilon}{2}E_n^0 = \frac{\epsilon}{2}\left(n + \frac{1}{2} \right) \hbar \omega,
\end{equation}
which is interesting considering that $\omega$ includes the original spring constant.

\section*{Problem 9.2}
A spin-$\frac{1}{2}$ degree of freedom is influenced by a magnetic field that has a large $z$-component and a small $x$-component such that the Hamiltonian is
\begin{equation}
\label{eq:magHam}
H = -\frac{B}{\hbar}S^z-\frac{g}{\hbar}S^x.
\end{equation}
The $x$-component of the field will be treated as a perturbation.

The unperturbed Schrödinger equation reads 
\begin{equation}
H\ket{n} = E_n^0\ket{n}.
\end{equation}
Employing Pauli matrices for convenience, one must find the eigenvalues of
\begin{equation}
H^0=-\frac{B}{\hbar}\frac{\hbar}{2}
\begin{bmatrix}
1 & 0 \\
0 & -1
\end{bmatrix}
= -\frac{B}{2}
\begin{bmatrix}
1 & 0 \\
0 & -1
\end{bmatrix}
\end{equation}
if
\begin{equation}
\ket{\uparrow^0} = 
\begin{bmatrix}
1 \\ 0
\end{bmatrix}, \quad
\ket{\downarrow^0} = 
\begin{bmatrix}
0 \\ 1
\end{bmatrix},
\end{equation}
then it is quite easy to see that the ground state energy eigenvalues must be
\begin{equation}
E_\uparrow^0 = -\frac{B}{2}, \quad E_\downarrow^0 = \frac{B}{2}
\end{equation}

\subsection*{a}

Now to find the change in energy and due to the perturbation Hamiltonian
\begin{equation}
H' = -\frac{g}{\hbar}\frac{\hbar}{2}
\begin{bmatrix}
0 & 1 \\
1 & 0
\end{bmatrix}
= -\frac{g}{2}
\begin{bmatrix}
0 & 1 \\
1 & 0
\end{bmatrix}
\end{equation}
The first order shift in ground state is 
\begin{equation}
E_\uparrow^1 = \bra{\uparrow^0} H'\ket{\uparrow^0} = -\frac{g}{2} 
\begin{bmatrix}
1 & 0
\end{bmatrix}
\begin{bmatrix}
0 & 1 \\
1 & 0
\end{bmatrix}
\begin{bmatrix}
1 \\ 0 
\end{bmatrix}=0,
\end{equation}
which means that there is \emph{no} first-order shift in ground state energy. You will get the same result for $\ket{\downarrow}$ and/or using $S^x = \frac{1}{2}(S^+ + S^-)$ as well.
\begin{equation}
E_\downarrow^1 = -\frac{g}{\hbar}\bra{\downarrow^0}S^x\ket{\downarrow^0} = -\frac{g}{2\hbar}\bra{\downarrow^0} (S^+ + S^-) \ket{\downarrow^0} = -\frac{g}{2} \bra{\downarrow^0} \ket{\uparrow^0} = 0
\end{equation}

\subsection*{c}

The perturbed first order eigenkets are found by way of the following formula
\begin{equation}
\ket{n} = \ket{n^0}+  \sum_{m} \frac{\ket{m^0}\bra{m^0}H'\ket{n^0}}{E_n^0-E_m^0} = \ket{n^0} + \ket{n^1}
\end{equation} 
the first order correction is given by $\ket{n^1}$. This gives us
\begin{align*}
\ket{\uparrow} &= 
\begin{bmatrix}
1 \\ 0
\end{bmatrix}
+
\frac{ 
\begin{bmatrix}
0 \\ 1
\end{bmatrix}
\begin{bmatrix}
0 & 1
\end{bmatrix}
\left(-\frac{g}{2}\begin{bmatrix}
0 & 1 \\
1 & 0
\end{bmatrix}\right)
\begin{bmatrix}
1 \\ 0
\end{bmatrix}
}{-\frac{B}{2}-\frac{B}{2}}
\\&= 
\begin{bmatrix}
1 \\ 0
\end{bmatrix}
+
\frac{ 
\begin{bmatrix}
0 & 0 \\
0 & 1
\end{bmatrix}
\left(-\frac{g}{2}\begin{bmatrix}
0 \\ 1
\end{bmatrix}\right)
}{-B}
\\&=
\begin{bmatrix}
1 \\ 0
\end{bmatrix}
+ 
\frac{g}{2B}
\begin{bmatrix}
0 \\ 1
\end{bmatrix}
= \ket{\uparrow^0} + \frac{g}{2B}\ket{\downarrow^0},
\end{align*}
similarly,
\begin{equation*}
\ket{\downarrow} = \ket{\downarrow^0} - \frac{g}{2B}\ket{\uparrow^0}
\end{equation*}

\subsection*{b}
Using the perturbed first order wave function one can calculate the second-order energy shift due to perturbation using the following formula.
\begin{equation}
E^2_n = \bra{n^0}H'\ket{n}.
\end{equation}
We get
\begin{align*}
E_\uparrow^2 &= \begin{bmatrix}
1 & 0
\end{bmatrix} 
\left(
-\frac{g}{2}
\begin{bmatrix}
0 & 1 \\
1 & 0
\end{bmatrix}
\right)
\left(
\begin{bmatrix}
1 \\ 0
\end{bmatrix}
+\frac{g}{2B}
\begin{bmatrix}
0 \\ 1
\end{bmatrix}
\right)
= -\frac{g^2}{4B} \\
E_\downarrow^2 &= \begin{bmatrix}
0 & 1
\end{bmatrix} 
\left(
-\frac{g}{2}
\begin{bmatrix}
0 & 1 \\
1 & 0
\end{bmatrix}
\right)
\left(
\begin{bmatrix}
0 \\ 1
\end{bmatrix}
-\frac{g}{2B}
\begin{bmatrix}
1 \\ 0
\end{bmatrix}
\right)
= \frac{g^2}{4B}
\end{align*}

\section*{Problem 9.3}
This is a problem illustrating both first-order non-degenerate and degenerate perturbation theory. Consider the two-dimensional harmonic oscillator with an extra bilinear term $gxy$, $g \in \mathcal{R}$.
\begin{equation}
H = \frac{p_x^2}{2m} + \frac{p_y^2}{2m} + \frac{1}{2}m\omega^2x^2 \frac{1}{2}m\omega^2y + gxy.
\end{equation} 
For $g=0$ the exact energy eigenstates are tensor products of one-dimensional harmonic oscillator states: $\ket{n_x,n_y}=\ket{n_x}\otimes\ket{n_y}$, where $n_x,n_y \in \{0,1,\dots \}$. Their energies are $E_{n_x,n_y}=\hbar\omega(n_x + n_y +1)$.

\subsection*{a}
The two lowest energies are
\begin{equation*}
E_{0,0} = \hbar\omega, \quad E_{1,0}=E_{0,1} = 2\hbar\omega,
\end{equation*}
corresponding to the eigenstates
\begin{equation*}
\ket{0,0}, \quad \ket{1,0}, \quad \ket{0,1}.
\end{equation*}
We see that the ground state is non-degenerative and the next-lowest energy level has a degeneracy of 2.

\subsection*{b}
If $g=0$ first-order non-degenerate perturbation theory can be used to compute how the ground state energy changes. The first-order energy shift will be
\begin{align*}
E_{00}^1 	&= \bra{0,0}H\ket{0,0} = \bra{0,0}qxy\ket{0,0} \\
			&=  g\frac{\hbar}{2}\frac{1}{m\omega}(\bra{0}\otimes\bra{0})(a_x^{\dagger}+a_x)(a_y^{\dagger}+a_y)(\ket{0}\otimes\ket{0}) \\
			&=  g\frac{\hbar}{2}\frac{1}{m\omega}(\bra{0}\otimes\bra{0})\left[(a^{\dagger}+a)\ket{0}\otimes(a^{\dagger}+a)\ket{0}\right] \\
			&= g\frac{\hbar}{2}\frac{1}{m\omega}(\bra{0}\otimes\bra{0})(\ket{1}\otimes\ket{1}) \\
			&= g\frac{\hbar}{2}\frac{1}{m\omega}\braket{0,0}{1,1} = 0.
\end{align*}

\subsection*{c}
By employing first-order degenerate perturbation theory one can find how the first excited energy splits up when $g$ is finite. The fundamental result of degenerate perturbation theory is the following formula.
\begin{equation}
\label{eq:degpet}
E_\pm^1=\frac{1}{2}\left(W_{aa} + "_{bb} \pm \sqrt{(W_{aa}-W_{bb})^2 + 4\abs{W_{ab}}^2} \right)
\end{equation}
where $W_{ij} = \bra{\psi_i^0}H'\ket{\psi}$. Setting $\ket{0,1}=\psi_a^0$ and $\ket{1,0}=\psi_b^0$ gives
\begin{align*}
W_{ab} 	&= \bra{0,1}H'\ket{1,0} \\
		&= \frac{g\hbar}{2m\omega}\bra{0,1}(a_x^{\dagger}+a_x)(a_y^{\dagger}+a_y)\ket{1,0} \\
		&= \frac{g\hbar}{2m\omega}\braket{0,1} = \frac{g\hbar}{m\omega}, 
\end{align*}
similarly
\begin{equation*}
W_{ba} = \frac{g\hbar}{2m\omega}, \quad W_{aa} = 0, \quad W_{bb} = 0.
\end{equation*}
Inserting into equation \ref{eq:degpet} yields
\begin{equation*}
E_\pm^1=\frac{1}{2}\left(0 + 0 \pm \sqrt{0 + \frac{4}{4}\frac{q^2\hbar^2}{m^2\omega^2}}\right) = \pm \frac{g\hbar}{2m\omega}
\end{equation*}

\subsection*{d}
Introducing the reflection operator
\begin{equation}
\label{eq:reflectionoperator}
R\ket{n_1}\otimes\ket{n_2}=\ket{n_2}\otimes\ket{n_1}
\end{equation}
For $g=0$ I will find eigenstates of $R$ that also are eigenstates of $H$, all with energies $2\hbar\omega$, 
id est they belong to the first excited energy level.

By setting
\begin{equation*}
\ket{0,1} = \begin{bmatrix} 0 \\ 1 \end{bmatrix} \quad \ket{1,0} = \begin{bmatrix} 1 \\ 0 \end{bmatrix}
\end{equation*}
then
\begin{equation*}
R = \begin{bmatrix}
0 & 1 \\
1 & 0 
\end{bmatrix}
\end{equation*}
$R$ is a unitary matrix and therefore has eigenvalues $\pm1$. This means that the eigenstates of $R$ can be
\begin{align*}
\begin{bmatrix} 1 \\ 1 \end{bmatrix} &=  \begin{bmatrix} 0 \\ 1 \end{bmatrix} + \begin{bmatrix} 1 \\ 0 \end{bmatrix} \\
\begin{bmatrix} 1 \\ -1 \end{bmatrix} &=  \begin{bmatrix} 1 \\ 0 \end{bmatrix} - \begin{bmatrix} 0 \\ -1 \end{bmatrix}
\end{align*}
Normalising these we end up with
\begin{equation}
\psi_\pm^0 = \frac{1}{\sqrt{2}}(\ket{1,0} \pm \ket{0,1})
\end{equation}

\subsection*{e}
With the ``good" states from the previous sub-problem, one can apply non-degenerate first order perturbation theory.
\begin{align*}
E_+^1 	&= \bra{\psi_+^0}H'\ket{\psi_+^0} \\
		&= \frac{g}{2}(\bra{1,0} + \bra{0,1})xy(\ket{1,0} + \ket{0,1}) \\
		&= \frac{g\hbar}{4m\omega}(\bra{1,0} + \bra{0,1})(a_x^{\dagger}+a_x)(a_y^{\dagger}+a_y)(\ket{1,0} + \ket{0,1}) \\
		&= \frac{g\hbar}{4m\omega}(\bra{1,0} + \bra{0,1})(\ket{0,1} + \ket{1,0}) = \frac{g\hbar}{2m\omega}
\end{align*}
\begin{align*}
E_-^1 	&= \bra{\psi_-^0}H'\ket{\psi_-^0} \\
		&= \frac{g}{2}(\bra{1,0} - \bra{0,1})xy(\ket{1,0} - \ket{0,1}) \\
		&= \frac{g\hbar}{4m\omega}(\bra{1,0} - \bra{0,1})(a_x^{\dagger}+a_x)(a_y^{\dagger}+a_y)(\ket{1,0} - \ket{0,1}) \\
		&= \frac{g\hbar}{4m\omega}(\bra{1,0} - \bra{0,1})(\ket{0,1} - \ket{1,0}) = -\frac{g\hbar}{2m\omega}
\end{align*}
We see that we get the same results using the ``good'' states for non-degenerate perturbation as for degenerate perturbation. Neat!

\end{document}