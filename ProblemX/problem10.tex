\documentclass[11pt]{amsart}

\usepackage{amsmath}
\usepackage{physics}
\usepackage[utf8]{inputenc}

\title[Problem Sheet 10]{Problem Sheet 10\\
		\large{FYS3110}}
		
\author[Winther-Larsen]{Sebastian G. Winther-Larsen}

\date{\today}

\begin{document}

\maketitle

\section*{Problem 10.1}
The Zeeman correction, when choosing the external field $\vb{B}_{ext}$ to lie along the $z$-axis, can be expressed by the following condensed formula.
\begin{equation}
\label{eq:condensedzeeman}
E_Z^1 = \mu_B g_J B_{ext} j_z,
\end{equation}
where
\begin{equation}
\label{eq:bohrmagneton}
\mu_B = \frac{e\hbar}{2m} = 5.788\times10^{-5}eV/T
\end{equation}
is the Bohr magneton, and
\begin{equation}
\label{eq:langeg}
g_J=1 + \frac{j(j+1) + \frac{3}{4} - l(l+1)}{2j(j+1)}
\end{equation}
is the Landé g-factor. Adding the fine structure equation
\begin{equation}
\label{eq:finestructure}
E_{nj} = -\frac{13.6eV}{n^2}\left[1 + \frac{\alpha^2}{n^2} \left(\frac{n}{j + \frac{1}{2}} -\frac{3}{4} \right) \right]
\end{equation}
to the Zeeman correction (equation \ref{eq:condensedzeeman}) yields an equation for total energy in presence of weak-field Zeeman effect
\begin{equation}
\label{eq:totalweakzeeman}
E_{nljj_z} = -\frac{13.6eV}{n^2} \left[1 + \frac{\alpha}{n^2}\left(\frac{n}{j + \frac{1}{2}} -\frac{3}{4} \right) \right] + \mu_B g_J B_{ext} j_z.
\end{equation}

For $n=2$

\end{document}