\documentclass{article}

\usepackage{amsmath}
\usepackage{physics}
\usepackage{bbm}
\usepackage{listings}

\title{Midterm ``Take home''-exam\\
		\large{FYS3110}}

\author{Unable to see candidate no}

\date{\today}

\begin{document}

\maketitle

\section{Spin-$1/2$ systems}

The following is given:
\begin{align*}
\hat{S}^2 = \hat{S}_x^2+\hat{S}_y^2+\hat{S}_z^2, &\quad
\hat{S}^{\pm}=\hat{S}_x \pm i\hat{S}_y \\
\ket{\uparrow} \equiv \ket{s=\frac{1}{2},m_s=\frac{1}{2}}, &\quad 
\ket{\downarrow} \equiv \ket{s=\frac{1}{2},m_s=-\frac{1}{2}} \\
\hat{S}^2\ket{\uparrow} = \hbar^2\frac{1}{2}\left(\frac{1}{2}+1\right)\ket{\uparrow}, &\quad
\hat{S}^2\ket{\downarrow} = \hbar^2\frac{1}{2}\left(\frac{1}{2}+1\right)\ket{\downarrow} \\
\hat{S}_z\ket{\uparrow} = \frac{\hbar}{2}\ket{\uparrow}, &\quad 
\hat{S}_z\ket{\downarrow} = -\frac{\hbar}{2}\ket{\downarrow} \\
[\hat{S}_x,\hat{S}_y]=i\hbar\hat{S}_z, \quad
[\hat{S}_y,\hat{S}_z]&=i\hbar\hat{S}_x, \quad
[\hat{S}_z,\hat{S}_x]=i\hbar\hat{S}_y 
\end{align*}

\subsection{}
\begin{equation*}
\hat{S}_z\hat{S}^+\ket{\downarrow} = \hat{S}_z\hat{S}_x\ket{\downarrow} + i\hat{S}_z\hat{S}_y\ket{\downarrow}
\end{equation*}
rewriting commutation relations
\begin{align*}
[\hat{S}_z,\hat{S}_x]=\hat{S}_z\hat{S}_x - \hat{S}_x\hat{S}_z =i\hbar\hat{S}_y  &\rightarrow
\hat{S}_z\hat{S}_x = i\hbar\hat{S}_y + \hat{S}_x\hat{S}_z \\
[\hat{S}_y,\hat{S}_z]=\hat{S}_y\hat{S}_z - \hat{S}_z\hat{S}_y =i\hbar\hat{S}_x  &\rightarrow
\hat{S}_z\hat{S}_y = \hat{S}_y\hat{S}_z - i\hbar\hat{S}_x,
\end{align*}
gives
\begin{align*}
\hat{S}_z\hat{S}^+\ket{\downarrow} &=(i\hbar\hat{S}_y+\hat{S}_x\hat{S}_z+i\hat{S}_y\hat{S}_z+\hbar\hat{S}_x)\ket{\downarrow} \\
&= \left(i\hbar\hat{S}_y -\frac{\hbar}{2}\hat{S}_x -i\frac{\hbar}{2}\hat{S}_y + \hbar\hat{S}_x \right) \ket{\downarrow} \\
&=\left(\frac{\hbar}{2}\hat{S}_x + i\frac{\hbar}{2} \hat{S}_y\right)\ket{\downarrow} = \frac{\hbar}{2}\hat{S}^+\ket{\downarrow}.
\end{align*}
This means that $\hat{S}^+\ket{\downarrow}$ is an eigenstate of $\hat{S}_z$ with eigenvalue $\hbar/2$.

\subsection{}
\begin{align*}
\hat{S}^-\hat{S}^+ &=(\hat{S}_x-i\hat{S}_y)(\hat{S}_x+i\hat{S}_y) \\
&= \hat{S}_x^2+i\hat{S}_x\hat{S}_y-i\hat{S}_y\hat{S}_x+\hat{S}_y^2 \\
&= \hat{S}_x^2+\hat{S}_y^2+i[\hat{S}_x,\hat{S}_y] \\
&= \hat{S}^2-\hat{S}_z^2-i\hbar\hat{S}_z
\end{align*}
This can be uses to compute the norm of $\ket{\psi_1} = \hat{S}^+\ket{\uparrow}$ and $\ket{\psi_2} = \hat{S}^+\ket{\uparrow}$.
\begin{align*}
\braket{\psi_1} &= \bra{\downarrow}\hat{S}^-\hat{S}^+\ket{\downarrow} = \bra{\downarrow}(\hat{S}^2-\hat{S}_z^2-\hbar\hat{S}_z)\ket{\downarrow} \\
&= \bra{\downarrow}\hbar^2\frac{1}{2}\left(\frac{1}{2} +1 \right)\ket{\downarrow} -\bra{\downarrow}\frac{\hbar^2}{4} \ket{\downarrow} +\bra{\downarrow}\frac{2\hbar^2}{4} \ket{\downarrow} \\
&= \frac{3\hbar^2}{4} -\frac{\hbar^2}{4}+\frac{2\hbar}{4} = \hbar^2 
\end{align*}
which means that $\norm{\ket{\psi_1}} = \hbar$.
\begin{align*}
\braket{\psi_2} &= \bra{\uparrow}\hat{S}^-\hat{S}^+\ket{\uparrow} = \bra{\uparrow}(\hat{S}^2-\hat{S}_z^2-\hbar\hat{S}_z)\ket{\uparrow} \\
&= \bra{\uparrow}\hbar^2\frac{1}{2}\left(\frac{1}{2} +1 \right)\ket{\uparrow} -\bra{\uparrow}\frac{\hbar^2}{4} \ket{\uparrow} -\bra{\uparrow}\frac{2\hbar^2}{4} \ket{\uparrow} \\
&= \frac{3\hbar^2}{4} -\frac{\hbar^2}{4}-\frac{2\hbar^2}{4} = 0
\end{align*}
which measn that $\norm{\ket{\psi_2}} = 0$.

\subsection{}

Phases are chosen sucht that the following relations hold
\begin{equation*}
\hat{S}^+\ket{\downarrow}=\hbar\ket{\uparrow}, \quad \hat{S}^-\ket{\uparrow}=\hbar\ket{\downarrow}.
\end{equation*}

Introducing a new state
\begin{equation*}
\ket{\phi}=\frac{1}{\sqrt{2}}(\ket{\uparrow}+e^{i\theta}\ket{\downarrow})
\end{equation*}
where $\theta$ is a real number. We wish to compute the ``uncertainty'' product $\sigma_{sx}^2\sigma_{sy}^2$ where
\begin{align*}
\sigma_{sx}^2 &= \bra{\phi}(\hat{S}_x-\bra{\phi}\hat{S}_x\ket{\phi} )^2\ket{\phi}  \\
\sigma_{sy}^2 &= \bra{\phi}(\hat{S}_y-\bra{\phi}\hat{S}_y\ket{\phi} )^2\ket{\phi}.
\end{align*}
First we need to find expressions for $\hat{S}_x$ and $\hat{S}_y$
\begin{align}
\label{eq:Sx}
\hat{S}^++\hat{S}^- &= (\hat{S}_x + i\hat{S}_y) + (\hat{S}_x-i\hat{S}_y) = 2\hat{S}_x 
\rightarrow \hat{S}_x = \frac{1}{2}(\hat{S}^++\hat{S}^-)\\
\label{eq:Sy}
\hat{S}^+-\hat{S}^- &= (\hat{S}_x + i\hat{S}_y) - (\hat{S}_x-i\hat{S}_y) = 2i\hat{S}_y 
\rightarrow \hat{S}_y = \frac{1}{2i}(\hat{S}^+-\hat{S}^-)
\end{align}
It will also make things easier to calculate $\hat{S}_x\ket{\uparrow}$, $\hat{S}_x\ket{\downarrow}$, $\hat{S}_y\ket{\uparrow}$ and $\hat{S}_y\ket{\downarrow}$. These values can be found using equations \ref{eq:Sx} and \ref{eq:Sy}.
\begin{align*}
\hat{S}_x\ket{\uparrow}		= \frac{\hbar}{2}\ket{\downarrow}& \quad
\hat{S}^2_x\ket{\uparrow}	= \frac{\hbar^2}{4}\ket{\uparrow}\\
\hat{S}_x\ket{\downarrow}	= \frac{\hbar}{2}\ket{\uparrow}& \quad
\hat{S}^2_x\ket{\downarrow}	= \frac{\hbar^2}{4}\ket{\downarrow}\\
\hat{S}_y\ket{\uparrow}		= -\frac{\hbar}{2i}\ket{\downarrow}& \quad
\hat{S}^2_y\ket{\uparrow}	= \frac{\hbar^2}{4}\ket{\uparrow}\\
\hat{S}_y\ket{\downarrow}	= \frac{\hbar}{2i}\ket{\uparrow}& \quad
\hat{S}^2_y\ket{\downarrow}	= \frac{\hbar^2}{4}\ket{\downarrow}\\
\end{align*}
We can begin on what is the real task at hand
\begin{align*}
\bra{\phi}\hat{S}_x\ket{\phi} 
&= \frac{1}{2}(\bra{\uparrow}+e^{-i\theta}\bra{\downarrow})\hat{S}_x(\ket{\uparrow}+e^{i\theta}\ket{\downarrow}) \\
&= \frac{1}{2}(\bra{\uparrow}+e^{-i\theta}\bra{\downarrow})\left(\frac{\hbar}{2}\ket{\downarrow}+e^{i\theta}\frac{\hbar}{2}\ket{\uparrow} \right) \\
&= \frac{\hbar}{4}\left(e^{i\theta}+e^{-i\theta} \right)\\ 
&= \frac{\hbar}{4}(\cos{\theta}+i\sin{\theta}+\cos{\theta}-i\sin{\theta}) \\
&= \frac{\hbar}{2}\cos{\theta} \\
\bra{\phi}\hat{S}_y\ket{\phi} 
&= \frac{1}{2}(\bra{\uparrow}+e^{-i\theta}\bra{\downarrow})\hat{S}_y(\ket{\uparrow}+e^{i\theta}\ket{\downarrow}) \\
&= \frac{1}{2}(\bra{\uparrow}+e^{-i\theta}\bra{\downarrow})\left(-\frac{\hbar}{2i}\ket{\downarrow}+e^{i\theta}\frac{\hbar}{2i}\ket{\uparrow} \right) \\
&= \frac{\hbar}{4i}\left(e^{i\theta}-e^{-i\theta} \right)\\ 
&= \frac{\hbar}{4i}(\cos{\theta}+i\sin{\theta}-\cos{\theta}+i\sin{\theta}) \\
&= \frac{\hbar}{2}\sin{\theta}
\end{align*}

\begin{align*}
\sigma_{sx}^2 
=& \bra{\phi}(\hat{S}_x-\frac{\hbar}{2}\cos{\theta})^2\ket{\phi} \\
=& \frac{1}{2}(\bra{\uparrow}+e^{-i\theta}\bra{\downarrow})(\hat{S}^2_x-\hbar\cos{\theta}\hat{S}_x+\frac{\hbar^2}{4}\cos^2\theta )(\ket{\uparrow}+e^{i\theta}\ket{\downarrow}) \\
=& \frac{1}{2}(\bra{\uparrow}+e^{-i\theta}\bra{\downarrow}) \\
&\left(
\frac{\hbar^2}{4}\ket{\uparrow}
+\frac{\hbar^2}{4}e^{i\theta}\ket{\downarrow}
-\frac{\hbar^2}{2}\cos{\theta}\ket{\downarrow}
-\frac{\hbar^2}{2} e^{i\theta}\cos{\theta}\ket{\uparrow}
+\frac{\hbar^2}{4}\cos^2{\theta}\ket{\uparrow}
+\frac{\hbar^2}{4}e^{i\theta}\cos^2{\theta}\ket{\uparrow} 
\right) \\
=&\frac{\hbar^2}{8}-\frac{\hbar^2}{4}e^{i\theta}\cos{\theta}+\frac{\hbar^2}{8}\cos^2{\theta}+\frac{\hbar^2}{8}e^{i\theta}\cos^2{\theta}+\frac{\hbar^2}{8}-\frac{\hbar^2}{4}e^{-i\theta}\cos{\theta} \\
=& \frac{\hbar^2}{4}-\frac{\hbar^2}{4}(\cos^2\theta+i\sin\theta\cos\theta) +\frac{\hbar^2}{8}\cos^2\theta+\frac{\hbar^2}{8}(\cos^3\theta+i\sin\theta\cos^2\theta)-\frac{\hbar^2}{4}(\cos^2\theta-i\sin\theta\cos\theta) \\
=& \frac{\hbar^2}{4}-\frac{3\hbar^2}{8}\cos^2\theta+\frac{\hbar^2}{8}(\cos^3\theta+i\sin\theta\cos^2\theta) \\
\sigma_{sy}^2
=& \bra{\phi}(\hat{S}_y-\frac{\hbar}{2}\sin{\theta})^2\ket{\phi} \\
=& \frac{1}{2}(\bra{\uparrow}+e^{-i\theta}\bra{\downarrow})(\hat{S}^2_y-\hbar\sin{\theta}\hat{S}_y+\frac{\hbar^2}{4}\sin^2\theta )(\ket{\uparrow}+e^{i\theta}\ket{\downarrow}) \\
=&\frac{1}{2}(\bra{\uparrow}+e^{-i\theta}\bra{\downarrow}) \\
&\left(
\frac{\hbar^2}{4}\ket{\uparrow}
+\frac{\hbar^2}{4}e^{i\theta}\ket{\downarrow}
+\frac{\hbar^2}{2i}\sin{\theta}\ket{\downarrow}
-\frac{\hbar^2}{2i} e^{i\theta}\sin{\theta}\ket{\uparrow}
+\frac{\hbar^2}{4}\sin^2{\theta}\ket{\uparrow}
+\frac{\hbar^2}{4}e^{i\theta}\sin^2{\theta}\ket{\uparrow} 
\right) \\
=&\frac{\hbar^2}{8}-\frac{\hbar^2}{4i}e^{i\theta}\sin{\theta}+\frac{\hbar^2}{8}\sin^2{\theta}+\frac{\hbar^2}{8}e^{i\theta}\sin^2{\theta}+\frac{\hbar^2}{8}+\frac{\hbar^2}{4i}e^{-i\theta}\sin{\theta} \\
=& \frac{\hbar^2}{4}-\frac{\hbar^2}{4i}(\sin\theta\cos\theta+i\sin^2\theta)+\frac{\hbar^2}{8}\sin^2\theta+\frac{\hbar^2}{8}(\sin^2\theta\cos\theta+i\sin^3\theta) +\frac{\hbar^2}{4i}(\sin\theta\cos\theta-i\sin^2\theta) \\
&=\frac{\hbar^2}{4}-\frac{3\hbar^2}{8}\sin^2\theta+\frac{\hbar^2}{8}(\sin^2\theta\cos\theta+i\sin^3\theta)
\end{align*}
And finally
\begin{align*}
\sigma_{sx}^2 \sigma_{sy}^2 
= \frac{\hbar^4}{64}(e^{i\theta}\sin^2\theta-3\sin^2\theta+2)(e^{i\theta}\cos^2\theta-3\cos^2\theta+2)
\end{align*}

\subsection{}
A system has three interacting spin degrees of freedom with the followin hamiltonian
\begin{equation}
H = \frac{J}{\hbar^2}(\vb{S}_1\cdot\vb{S}_2 + \vb{S}_2\cdot\vb{S}_3 + \vb{S}_3\cdot\vb{S}_1)
\end{equation}
where $J$ si a positive number with units of energy. The spin operators are $\vb{S}_1\equiv \vb{S}\otimes\mathbbm{1}\otimes\mathbbm{1}$, $\vb{S}_2\equiv \mathbbm{1}\otimes\vb{S}\otimes\mathbbm{1}$ and $\vb{S}_3\equiv \mathbbm{1}\otimes\mathbbm{1}\otimes\vb{S}$, where $\vb{S}=(S_x,S_y,S_z)$. A general state of this three-spin system is a linear combination of product states $\ket{m_{s1}m_{s2}m_{s3}}\equiv\ket{m_{s1}}\otimes\ket{m_{s2}}\otimes\ket{m_{s3}}$ where $m_{si}$ is hte spin-$z$ quantum number of spin number $i$, either up ($\frac{1}{2}$) or down ($-\frac{1}{2}$). For example: the product state $\ket{\uparrow\downarrow\uparrow}$ is a state where spin number one is in state $\ket{\uparrow}$, spin number two is in state $\ket{\downarrow}$ and spin number three is in state $\ket{\uparrow}$.

$\vb{S}_1\cdot\vb{S}_2$ can be expressed in terms of $S_1^+$, $S_1^-$, $S_2^+$, $S_2^-$, $S_1^z$ and $S_2^z$. First we have
\begin{equation*}
\vb{S}_1\cdot\vb{S}_2=S_1^xS_2^x + S_1^yS_2^y + S_1^zS_2^z
\end{equation*}
where
\begin{align*}
S_1^xS_2^x= \frac{1}{4}(S_1^+S_2^+ +S_1^+S_2^- + S_1^-S_2^+ + S_1^+S_2^+)\\
S_1^yS_2^y=-\frac{1}{4}(S_1^+S_2^+ -S_1^+S_2^- - S_1^-S_2^+ + S_1^+S_2^+)
\end{align*}
then
\begin{equation*}
S_1^xS_2^x + S_1^yS_2^y = S_1^+S_2^-
\end{equation*}
assuming that the lowering and raising operators of different spins commutes\footnote{Ladder operator for same spin/state commute: $[S^+,S^-]=(S^x+iS^y)(S^x-iS^y)-(S^x-iS^y)(S^x+iS^y)=S_x^2+S_y^2-S_x^2-S_y^2$=0}, i.e. $[S_i^+,S_j^j]=0$. We end up with
\begin{equation}
\vb{S}_1\cdot\vb{S}_2 = S_1^+S_2^- + S_1^zS_2^z
\end{equation}
if the ladder operators does not commute then
\begin{equation*}
S_1^xS_2^x + S_1^yS_2^y = \frac{1}{2}(S_1^+S_2^- + S_2^+S_1^-)
\end{equation*}
and we end up with
\begin{equation}
\vb{S}_1\cdot\vb{S}_2 = \frac{1}{2}(S_1^+S_2^- + S_2^+S_1^-) + S_1^zS_2^z
\end{equation}
which seems more reasonable.

Comtuting $H\ket{\uparrow\downarrow\downarrow}$ should now be quite straight-forward.
\begin{align*}
H\ket{\uparrow\downarrow\downarrow} = 
\frac{J}{\hbar^2} \big(&\frac{1}{2} (S_1^+S_2^- + S_2^+S_1^-)\ket{\uparrow\downarrow\downarrow} + S_1^zS_2^z\ket{\uparrow\downarrow\downarrow} \\
+&\frac{1}{2}(S_2^+S_3^- + S_3^+S_2^-)\ket{\uparrow\downarrow\downarrow} + S_2^zS_3^z\ket{\uparrow\downarrow\downarrow} \\
+&\frac{1}{2}(S_3^+S_1^- + S_1^+S_3^-)\ket{\uparrow\downarrow\downarrow} + S_3^zS_1^z\ket{\uparrow\downarrow\downarrow} \big)\\
=&\frac{J}{\hbar^2}\big(
\frac{\hbar^2}{2}\ket{\downarrow\uparrow\downarrow} 
\frac{\hbar^2}{2}\ket{\downarrow\downarrow\uparrow} 
\frac{\hbar^2}{4}\ket{\uparrow\downarrow\downarrow} 
\big) \\
=&J(
\frac{1}{2}\ket{\downarrow\uparrow\downarrow} 
\frac{1}{2}\ket{\downarrow\downarrow\uparrow} 
\frac{1}{4}\ket{\uparrow\downarrow\downarrow})
\end{align*}

This result is confirmed by the python script in appendix \ref{app:updndn}.

%% --------------- PROBLEM 1.5 -------------- %%
\subsection{}

%% --------------- PROBLEM 1.6 -------------- %%
\subsection{}

%% --------------- PROBLEM 1.7 -------------- %%
\subsection{}

%% --------------- PROBLEM 1.8 -------------- %%
\subsection{}
At time $t=0$ the system is in state $\ket{\uparrow\downarrow\downarrow}$. After som time $t$ the system will be in state $\hat{U}(t,t_0)\ket{\uparrow\downarrow\downarrow}$, where $\hat{U}(t,t_0)$ is the time evolution operator (or propegator). The propagator satisfy three important properties. First, it does nothing when $t=0$
\begin{equation}
\lim_{t\to t_0}\hat{U}(t,t0)=1.
\end{equation}
Second, it is unitary ($\hat{U}^{\dagger}\hat{U}=1$), and as a consequence preserves the norm of the states
\begin{equation}
\braket{\psi} = \braket{\psi(t)} = \bra{\psi(t)} \hat{U}^{\dagger}(t,t_0){U}(t,t_0) \ket{\psi(t)}
\end{equation}
Third, it satisfies the composition property
\begin{equation}
\hat{U}(t_2,t_0)=\hat{U}(t_2,t_1)\hat{U}(t_1,t_0)
\end{equation}

One can see from the simplest form of Scrhödinger's equation that the Hamiltonian $H$ generates the time evolution of quantum states. if $\ket{\psi(t)}$ is the state of the system at time $t$, then
\begin{equation}
H\ket{\psi(t)}=i\hbar\frac{\partial}{\partial t}\ket{\psi(t)}.
\end{equation}

Given the state at some initial time ($t=0$) one can solve Schrödinger's equation in order to obtain the state at any subsequent time. Particularly, if $H$ is independent of time, then
\begin{equation}
\ket{\psi(t)}=e^{-iHt/\hbar}\ket{\psi(0)}.
\end{equation}
This exponential operator is usually defined by the corresponding power series. 
\begin{equation}
U(t) = \sum_{k=0}^{\infty}\frac{1}{k!}\left(\frac{it}{\hbar} \right)H^k.
\end{equation}

If a spin system is initially in state $\ket{\uparrow\downarrow\downarrow}$ at $t=t_0=0$ then the expression $\abs{\bra{\uparrow\downarrow\downarrow}\hat{U}(t,0) \ket{\uparrow\downarrow\downarrow}}^2$ gives the probability that the system is still in that state.

\section{}
TAYLOR THEN TRUNCATE?

\pagebreak
\begin{appendix}
\section{Numerical computation of $H\ket{\uparrow\downarrow\downarrow}$}
\label{app:updndn}
\lstinputlisting[language=Python]{scripts/tensors.py}

\end{appendix}

\end{document}
